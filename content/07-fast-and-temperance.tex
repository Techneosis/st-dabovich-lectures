\chapter{Fast and Temperance}

Man, having received his present being, 
consisting of a visible body and intellectual,
immaterial soul, is a being complex. 
But the nature and worth of both the just 
named parts are not of equal value. The body
is made as an instrument that is moved by the 
order of a ruler; the soul is designed to govern 
and command it, as the superior of an inferior. 
The soul, receiving from the intellect and reason
the means by which it makes distinctions, 
may, possessing such a quality of distinction, 
separate the truly beautiful from its common 
imitation; it may perceive God as the Creator 
and Designer, not only of that which is underneath
our feet and received by our senses, but 
that, also, which is hidden from the eyes, and 
of which the immaterial mind may contemplate, 
having the power of imagination at its command. 
Practicing, as the godly one, in righteousness 
and virtue, she aspires unto divine wisdom, and, 
obeying its laws and commands, withdraws as 
much as possible from the desires of the flesh, 
comes nearer to God, and strives by all its 
strength to ally itself with the good. The 
particular and most important object of this 
sacred philosophy is temperance; as it is the 
mind, which is not disturbed, but free of all influences
of pollution, arising from the stomach 
or other senses, that has a continual action and 
contemplates the heavenly, the things pertaining
to its own sphere. 

And so it behooves us, the lovers of all things 
pure, the lovers of the word of God, yea -- even 
Christians, to love the present time, which our 
holy church has set apart for a special opportunity
of obtaining greater grace in the sight 
of God. We should hail with joy such an opportunity!
The time I refer to is the Advent 
Lent. We should love this fast as the teacher 
of sobriety, the mother of virtue, the educator 
of the children of God, the guardian of the 
unruly, the quiet of the soul, the staff of life, the 
peace that is firm and serene. Its importance 
and strictness pacifies the passions, puts out 
the fire of anger and wrath, cools and quiets the 
agitation produced by over-eating. And, as in 
summer time, when the sweltering heat of the
sun hangs over the ground, the northern breeze 
proves a blessing to the sufferers, scattering the 
closeness by its pleasant coolness, so does likewise
fast, destroying the overabundance of heat 
in the body, which is caused by gluttony. Proving
to be of so much benefit to the soul, Lent 
brings the body no less benefit. It refines the 
coarseness of matter, releases the body of part 
of its burden, lightens the blood vessels that 
are often ready to burst with an overflow of 
blood, and prevents them becoming clogged, 
which may happen as easily as it occurs with a 
water pipe, that, when being forced to maintain 
the abundance of water pressed into it by a 
powerful machine, bursts from the pressure. 
And the head feels light and clear when the 
blood-vessels do not nervously beat, and the 
brain does not become clouded by the spreading 
of evaporations. Abstinence gives the stomach 
ease, which relieves it from a forced condition
of slavery, and from boiling like a boiler, 
working with a sickly effort to cook the food it 
contains. The eyes look clear and undimmed, 
without the haze that generally shadows the 
vision of a glutton. The activity of the limbs 
is stable, that of the hand firm; the breath is 
regular and even, and not burdened by pent-up 
organs. The speech of him who fasts is plain 
and distinct; the mind is pure, and then it is 
that the mind shows forth its true image of 
God, when, as if in an immaterial body, it 
quietly and undisturbedly exercises the functions
belonging to it. The sleep is quiet and 
free from all apparitions. Not to extend 
unnecessarily, we may sum up by saying that 
fast is the common peace of the soul and body. 
Such are the beneficent results of a temperate 
life; and such are the precepts of a Christian 
life. It is a law of the Holy Church, which 
prescribes that we should fast during the Lenten 
season. 

Do you not know that angels are the constant 
watchers and guardians of those that fast, just 
as the demons, those very friends of greasy 
stuffs, those lovers of blood and companions of 
drunkards, are the associates of those that give 
themselves up to debauchery and orgies during 
such a holy time as lent? The angels and saints, 
as also the evil spirits, ally themselves with 
those they love, they become related with that, 
which is pleasing to them. Every day in our 
life God points out a lesson to us concerning 
the eternal life, but we very seldom heed it; 
in a word, we generally don't care! Oh, is this 
not terrible to think of? And yet no one man 
will deliberately, so to speak, attempt to slight 
the Almighty Creator, no one who is capable of 
using his understanding in the very least degree.
But yet, beloved brethren, we do it! 
We, day after day, in our worldly habits unconsciously
say: ``I don't care!'' Have we a right 
to do anything at all unconsciously, when He, 
in whose hand the very breath of our life 
flutters as a very weak, little thing, when He, 
I say, bestowed upon us this conscience? Over 
and over again we dare to directly disobey 
God's commands. It is a terrible thing to fall 
into the hands of the Living God. But the 
Lord of Hosts is long-suffering, and to repentant
Christians He is the Father of Mercies. 
Yet it behooves us, Christians, to zealously 
watch every step we take, to be sure that we are 
walking in the path, that our Holy Church not 
only pointed out, but, as it were, even cut out 
for us by the stream of martyr's blood, by the 
wisdom of the Holy Ghost abiding in the 
sainted bishops of the universal councils, the 
night labor of praying and fasting fathers, and 
a host of pure, self-sacrificing, obedient women, 
such as Mary, Thekla, Barbara, Makrina. The 
church says that in time of Lent we must fast,
and we should not disobey, because our Holy 
Church is the Church of God, and she tells us 
what God Himself wills that we should do. If 
we have all the learning of the nineteenth century,
it will appear as a blank before the simple 
words of the church, spoken in the power of the 
Spirit of God. We can not, and we have no 
right (for who gave us such a privilege), to 
excuse ourselves. We are with good intention,
in simplicity of heart to obey the commandments
of the church, and not worry about 
adapting ourselves to the ways of the church, 
for when we obey with our whole heart, with a 
strong desire to fulfill the holy commandments,
then our Holy Mother Church adapts 
herself to the weakness of her faithful children. 
But let us turn back to the lesson pointed out 
for us. We may every day learn a new lesson 
about the next life, which is of so much importance,
that the examples in this life are inexhaustible.
Look around and observe. In this 
instance look into the kingdom of animals and 
birds. See the clean dove hovering over places 
that are clean, over the grain field, gathering 
seed for its young. Now look at the unsatiated 
raven, flapping its heavy wings around the meat 
market. And so we must strive to love a
temperate life, that we may be beloved by 
angels, and hate all unnecessary luxury, so as 
not to fall with it into communion with demons.

Let us return with our memory to the commencement
of our race, and experience will 
testify to that which we sometimes make light 
of. The law of fasting would not be given to 
us, had not the law of the first abstinence been 
transgressed. The stomach would not be 
named as an evil-minded thing, had not the 
pretext for pleasure entailed after it such consequences
of sin. There would be no need of 
the plow and the laboring oxen, the planting of 
seed, the watering shower, the mutual change of 
the seasons of the year, the winter binding in 
fetters and the summer opening up all things. 
In a word there would be no need of such 
periodically repeating toil, had not we, through 
the mistaken pleasure of our first parents, condemned
ourselves to this round of labor. Yet 
we were on the way of leading another kind 
of life, in comparison with what we see now, 
and which we hope to regain once more, when 
we are liberated from this life of passion by 
the resurrection. Such is the mercy of God's 
condescension towards us, that we should be
again restored to the former dignity, which we 
had enjoyed through His love to man, and 
which mercy we did not carefully keep. Fast 
is a type of the future life, an imitation of the 
incorruptible existence. There are no feastings
and sensual gratifications over there. 

Do not flee from the difficulty of fast, but set 
up hope against the trial, and you will obtain 
the desired abstinence from food. Repeat to 
yourself the words of the pious: ``Fast is 
bitter, but paradise is sweet; thirst is tormenting,
but the spring, from which he who drinks 
will thirst never again, is at hand.'' The body 
is importunate, but the immaterial soul is 
much stronger -- strength is dead, but nigh is 
the resurrection. Let us say to our much craving
stomach what the Lord said to the tempter: 
\textit{Man shall not live by bread alone, but by every 
word of God}. Fast is not hunger, but a little 
abstinence from food, not an inevitable punishment,
but a voluntary continence, not a servile 
necessity, but a free selection of the wise. 
Pray and you will be strengthened; call, and a 
prompt helper will come to your assistance.

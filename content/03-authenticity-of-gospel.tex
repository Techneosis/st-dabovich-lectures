\chapter{The Authenticity and Truthfulness of the Gospel}

As you are aware, we have the gospel of our Lord Jesus Christ in the four books of the holy evangelists: Matthew, Mark, Luke, and John. Whole libraries of books sprung up, as it were, from under the pens of the most eminent scholars of the world in defense of the authenticity of our accepted gospel. Volume upon volume may, and in fact are, still being written, in proof of the truthfulness of the exposition we have of the work and teaching of our Lord Jesus Christ as laid down in the gospel. Evemn the heathen with their dark histories and mysterious traditions have contributed to explain from more than one side \textit{the hope of the nations, the glory of Israel, the light of the world}. If you are truly educated and take interest in this special line of study, you can freely make these investigations for yourself.

Unfortunately some who are affected with unbelief undertake the investigation of the gospel, and the narratives about the miracles of Jesus Christ, with the express purpose, not of learning the truth, but in order to denounce them. These few persons, (whose minds in every case were not intended by nature to be critical) are predetermined and have their minds previously biased with a false philosophy, so that, according to their fixed logic, there can be no miracles; therefore they must be overturned by all possible means, and they have used everything within reach of their power against them, but they could gain no results.

Some of the unbelievers in miracles acknowledge that the gospels were written by the apostles themselves, or from the words of the apostles, as the church believes, while others contend that the gospels, although they bear up the names of the apostles and their disciples, were not written by themselves, but by others, who lived later.

If the apostles themselves have written the gospels and the described the miracles of Jesus Christ, then the unbelievers may yet have these two explanations for doing away with miracles. First--the apostles have agreed among themselves to preach and to write falsehood. But it cannot be possible that the apostles should alone agree to do such a thing, because many Christians who lived in their time would also have to acknowledge and preach the untruth. They could not attempt tsuch a daring falsity, because they would be accused by the contemporaries of Jesus Christ; and why should they agree to such a thing, when--for their preaching Christ--nothing else but sorrow awaited them on earth? Besides this God would punish them in the future life for the falshood--especially in such an important matter. If we only in this life hope in Christ, then we are the most unfortunate of all people--thus writes St. Paul the Apostle to the Corinthians (1 Cor. xv: 15, 19). Secondly--some other unbelievers contended that the apostles did not understand the works of Jesus Christ, and that which was not miraculous, but natural, yet remarkable, in the life of Christ, they received as the miraculous. Nevertheless, the healing of the blind, the deaf, the lame and other incurables, even at an invisible distance, only by a word; the changing of the water into wine, the walking upon the sea, the feeding of five thousand people with five loaves, the raising of the dead, His own resurrection, and the ascension into heaven--these are such deed, which nohow can be explained in a natural way, and it is impossible to be mistaken in such like miracles. This ''mistaken explanation'' has been cast aside by the unbelievers themselves.

What has been mentioned thus far is not itself the defense of the Gospel, for you must understand that no human fortifications are necessary to the Divine Truth of the Almighty. We have been simply reminded that proofs of the authenticity of the gospel do exist, and that they are the expression of human thought and energy in man's effort to collect his gellow beings and be at one with Christ and God. Although the Almighty \textit{maketh His angels spirits, and His servants a fiery flame}, yet for our benefit in His love He rewards our efforts with His assistance, and bestows our outreaching arm with the sanctity of Divine authority. It is in this way that we have a most sure defense for Divine things in our human arguments--when they are set forth with love for the salvation of our neighbor.

The proofs of the truthfulness of the Gospel are these:
\begin{enumerate}
    \item The large number of foretypes in the Old Testament, and the prophecies of the Patriarchs and Prophets during 5000 years before the birth of our Lord Jesus Christ, both of which are so distinct and connected with historical facts well known to the whole world. These apply directly to the person of Jesus Christ the Messiah and also to our times, i.e. of the New Testament. A careful examination of the proofs under this single head is sufficient to convince a sincere seeker of the truth to embrace the gospel and become a Christian.
    \item Second come the testimonies contained in all the books of the New Testament. Besides the four Evangelists there were others who labored in putting the Word of God into writing; name'y, the Apostles Sts. James, Peter, Jude and Paul. The harmony throughout these Scriptures is indeed marvelous. It is the practiced reader alone who knows how to appreciate the beauty of God's power operating in so many different characters, at different places, in different times and under different circumstances.
    \item This proof of the truthfulness of the gospel is that influence, which only a subjective analysis of the individual life can disclose. You hear it in church in the hymns sung by the choir. Sometimes children see it in the face of their parents and feel it in the presence of their teachers. The penitent criminal speaks of it within the walls of a prison. The Christian soldier glories in it while he falls bleeding. Sometimes the ruler obeys its influence. The mechanic, the mechant likewise pass by its way. The poor widow puts her trust in it when she reads the Bible while surrounded by her hungry children. The dying sleep reposing in the gospel of Our Lord Jesus Christ.
    \item Another proof in defense of our subject is the history of the world. The history of each cilvilized nation, every legal tribunal of justice, all of the renowned universities of the earth, all of the societies and communities which have a strong hope for ultimate moral progress, the present condition of the family, the advancement of womanhood, the fine critical arguments of the politician, the pride of universal literature, the aciom of science, the investigation of the past, the examination of the present, the cheerful hopefulness of the serious and busy foundation builders of the future, even the glory of music and art, in all this we see traces of the gospel of Jesus Christ. Moreover it is acknowledged by the world, as plain as black and white, that the teaching of the New Testament is becoming rapidly more and more the inseparable companion to sober, thinking mankind. Thus we have a real objective evidence, moulding history before our very eyes, to the proof of the truthfulness of the Gospel. Long hours of discussion might be devoted to each one of these four witnesses of the Gospel separately. But you cannot bear with them. If I should undertake the gigantic task I would not be able to finish it.
    \item There is yet a fifth argument in favor of Christians. Although the last, it is by no means the least of them. This division contains the testimony of eye-witnesses and their correspondence with contemporaries, and other literary monuments of the first centuries of our era. Upon such important evidence we shall dwell now for a few moments.
\end{enumerate}

In the libraries of Europe, as well as in the libraries of the Eastern churches, such as Constantinople, Jerusalem, Antioch, Alexandria, Syria, Egypt, Georgia, Armenia, Mt. Athos, Servia, and also Russia, not a few of the manuscript gospels of ancient times have been preserved to our day. Some of them belong to the fourth century (i.e. They are almost 1600 years old). They are the same gospel that we have today. They describe the same miracles that we know of, and as our sacred scriptures have been copied from these books, so have they been in turn copied from books which were carefully and even jealously guarded--as we learn from immutable history--of the third, then of the second century, and the original books of the holy Apostles and their celebrated companions. The works of the renowned and learned Origne, who wrote his commentaries on the gospels of Matthew, Luke and John in the third centruy, and in which the text of the same gospels is almost completely contained. Origen flourished in the first half of the third century, and besides this we have the writings of many fathers of the church and other Christian workers belonging to the third century, which contain a great many quotations from the gospels that agree with outs, and concerning the miracles of Jesus Christ. Therefore there can be no doubt that our Gospel is the the same which was read in the third century.

The same must be said of the second half of the second century, after the birth of Christ. We have such works and publications of fathers and Christian writes belonging to this period, in which we find many places of the gospels. We will point to the ten books against heresy, which belong to the reverend martyr Irenaeus, the Bishop of Lyons. These books were written 150 years after the ascension of our Lord, and the unbelievers themselves do not doubt that Irenaeus is the author of them. In these books there are 400 quotations taken from the four gospels, and they come in the same order in which we read them today. The texts mentioned refer to the same miracles of Jesus Christ. There are 80 quotations taken from the gospel of St. John. St. Irenaeus speaks of the evangelists by name, and he gives historical information in regard to the gospels, which they have written. He affirms that the church, which was spread widely in his time, held no more nor less than the four gospels, as there are four sides of the earth, four winds, as the Lord appeard to the prophet Hezekiel, sitting upon four chrubim. Many places from the gospels are mentioned also by contemporary writers, such as Tertulianus, Clement, of Alexandria, and others, so that, upon this literature, which ahs come down to us from these Christian men of the second half of the second century, and which belongs to them without a doubt, the contents of the four gospels known to us could be restored. It is acknowledged by students that in the last half of the second century there has been a translation into the Syrian from the Greek language, not only of the four gospels, but also of other books of the New Testament known to us.

We have positive witness in the writings of the disciples of the apostles or their contemporaries, who decidedly ascribe the gospels to the authorship (of course from the human side) of the apostles, or to the disciples of the apostles, as is recognized by the whole church.

The first witness is St. Polycarp, the reverend martyr, a disciple of St. John the Divine, and other apostles. He was Bishop of Smyrna in Asia. An epistle of St. Polycarp to the Philippians has been handed down to our day. In it he says: ''Any one who does not confess that Jesus Christ came in the flesh, he is antichrist. In the epistles of John the apostle and evangelist it is said: \scripture{Every spirit which confesseth not Jesus Christ as having come in the flesh, the same is not of God, and the same is antichrist.} Thus, in the first place, this disciple of St. John the Divine speaks of him as the evangelist, consequently John has written the gospel; and secondly, to quote literally from the first general epistle of St. John, while this epistle is prefectly alike in style and thought to the fourth gospel, it is to verify the apostle John as the writer of this gospel. We see in the writings of St. Polycarp indications of the gospel of Matthew and other books of the apostles. The above mentioned epistle of St. Polycarp to the Philippians is acknowledged as originally belonging to him by the most reasonable of learned investigators. That St. Polycarp has written this epistle to the Philippians, to this his disciple testified, the above mentioned reverend martyr Irenaeus, and it is difficult to conceive how the genuine epistle of St. Polycarp could have been mutilated or lost and a false one spread abroad in its place. The epistle of St. Polycarp from the time since it was written has been read in the churches of Asia at divine service during the first centuries. It was read even in the fourth century.

Not alone through his writings, but by his long life and his death, which was that of a martyr, does St. Polycarp still more testify to the truth of the gospel.

By the act of God's providence, not onle the apostle John, but also some of the disciples of the apostles, lived a long life. St. Irenaeus, that great pillar of the structure of proofs, pointing out the original gospels, testifies that John the Evangelist lived to the time of the Emperor Trajan, who ascended the throne in 98 A. D., and died in 117, ocnsequently this apostle either died in the last two years of the first, or in the beginning of the second century after Christ. This testimony of St. Irenaeus, most accurate in itself, as he was the disciple of Polycarp, who was a contemporary and disciple of St. John, yet it is confirmed by many other ancient witnesses, who set the date of the death of John the Divine in the beginning of the second century.

The reverend martyr Polycarp, the disciple of the apostles, and especially of St. John, lived to be about 110 years. He died in 167 A. D. The disciple of Polycarp, St. Irenaeus, who was bishop of Lyons, in the country now called France, was put to death for preaching Christ in 202.

The gospels of Matthew, Mark and Luke could not be false and appear and be accepted by all the churches before the close of the first century, as some unbelievers say, and likewise the same of the Gospel of John about the year 150, because the immediate disciples of the apostles would not permit an error, and we know that St. Polycarp was the leader of all the churches in Asia until his martyrdom in 167. It was about 160 A. D. when Polycarp went to Rome, during the time of Bishop Anecetas, on account of a dispute concerning the time of celebrating Easter (or the resurrection of our Lord), and therefor he know the condition of the churches of the west, as well as of the east. If a difference of time in celebrating Easter Sunday was the cause of such a warm dispute, then what severe quarrels would take place if some of the churches had perchance accepted false gospels.

Yet there is not in all the works of the fathers and the writers of the second and following centuries so much as the hint of a doubt in the christian churches, each independany of the other, concerning the authenticity and correctness of the four gospels. The disciple of Polycarp, St. Irenaeus, who had our four gospels, was also acquainted with the condition of the churches of both the east and west; for in the east he was educated, and in the wwest he died a bishop. Ot wmust be remembered that both men, St. Polycarp and St. Irennaues, ended their life by a martyr's death for preaching Jesus Christ as the Son of God and the Savior of the world. St. Polycarp decided to be burned upon a pile rather than to renounce Christ.

We repeat that St. John the Divine had other disciples, but who of course did not live so long as Polycarp. Nevertheless they could have prevented during the first half of the second century the spreading of any gospels of Jesus Christ and His miracles, which might not have agreed with the teachings of the apostles, and particularly of St. John, with whom they have been in close relations. History has deposited for our sake a considerable portion of a letter written by St. Irenaeus to his friend, one Florinus, in which he says of a certain heresy: ''Thou has not been taught thus by the Preesbyters who preceded us, and who listened to the apostles personally.'' From this we may conclude that Irenaeus and his contemporaries have in their youth studied not alone by St. Polycarp, but also by other disciples of the apostles.

We may add more facts in connection with the testimony of Irenaeus. He was a co-laborer with and in the episcopate a successor of the revered martyr Pothinus, the Bishop of Lyons, in the country of the Gauls, the present France.

Who was St. Pothinus? According to a most ancient tradition of the Church of Lyons, he was a native of Asia, a disciple of Polycarp, and even of the apostles themselves. In the first half of the second century he came to Lyons and organized a church, for which he was ordained a bishop. Fifteen years after his arrival in Lyons, upon his request there were sent to his assistance several men from the east, who were qualified to preach the gospel, and among whom was the learned St. Irenaeus. St. Irenaeus became a presbyter in Lyons. In 177 St. Pothinus died in prison during a persectuion of the christians. The year of his death, 177 A.D., and his age are clearly certified. In the accounts of this age of martyrs, recorded by an eye-witness and preserved in the history of Eusebius, it is said of him that he was more than 90 years old at the time of his death. Therefore they concluded that he was born in 86 A. D., and he could have seen the apostle John. There is no doubt whatever that he was a contemporary, not only of St. Polycarp, but also of many other apostolic men, who were disciples of St. John the Divine.

He could have known men who have seen other apostles, that died earlier than St. John, and who labored in preaching the Gospel in Asia, for instance: St. Philip one of the twelve, who suffered in Hieropolis during the time of the Emperor Domitian (81-96). Thus, Irenaeus could have received in Lyons likewise--of St. Pothinus, who was much older than him in age--accurate information concerning the eastern apostolic churches, also of the holy gospels, and especially of the Gospel of St. John.

Another witness, like St. Polycarp, of the truthfulness of the Gospel is Papius, the Bishop of Hieropolis. He is known as the disciple of the disciples of the Lord. He died about the year 120, earlier than Polycarp. In his work known as the \textit{Five Books of Explanations of the Words of the Lord,}\footnote{These original documents have been discovered in Mossoul. The Patriarch of Antioch has recently brought them to Paris.} part of which are quoted by Eusebius, the historian, he says--on the testimony of the apostles of Jesus Christ--that the evangelist Matthew has written the words of the Lord in the Hebrew language, and Mark has written from the dictation of the apostle Peter of what the Lord did and taught.

Later on another portion of the writings of Papius has been discovered, which proves that he was acquainted with the Gospel of John. Besides this, the historian Eusebius, who had read  the writings of Papius, affirms that Papius took as testimony, words from the first epistle of John the Divine; and as this epistle in word and thought is exactly the same as the gospel of John, it verifies the fact that John was the writer of the gospel.

A contemporary of Polycarp and other apostolic men was St. Justin, the martyr and philosopher; he was profoundly educated and had written many works, but many of which, unfortunately, have not reached us. Two of his works have come down to us; they are apologies in defense of persecuted christians, one of which was handed to the Roman Emperor Antoninus Pius about 150 A.D., the other after some time was presented to the senate of Rome. The martyr Justin has also put to writing his dialogue with the learned Jew Triphonus in defense of christianity. What is important for us is the fact that the unbelievers could find nothing to say against the authenticity of the writings of St. Justin, the philosopher, and also the fact that he quotes many places from the gospels known to us; from the gospel of Matthew, beginning from the first chapter up to the last, there are fifty quotationsl from the gospel of Luke, about twenty; from the gospel of John, more than fifteen places, and he also makes mention of another gospel (i.e. of Mark.) He gives the proper name of \textit{evangelia} to the gospel, and also mentions them as the \textit{the remembrances or memory notes of the apostles and their companions.}

To return again to witness concerning the evangelist John, it is stated as certain that he spent the last years of his life in Ephesus; here he died and was buried as Policratus, the Bishop of Ephesus, who lived in the last part of the second century, writes in his letters to Victor, the Bishop of Rome; ancient literature has saved the contents of this letter. Moreover the Church Universal has always recognized this fact.

As St. Polycarp had for his disciple the reverend-martyr Irenaeus, a great witness for the evangelical truth, so had the holy martyr Justin a disciple in the person of Tacian, a witness of the four gospels. He was a learned pagen, who studied ancient philosophy, but not having found the truth, he turned himself to the Christian Church. He was already a full grown man when he became a pupil of St. Justin in Rome, after which he continued in close friendship with him. After the death of Justin he went to the east, and in Syria, unfortunately absorbed in meditation, he attached too weighty imporance in his own reasoning and fell in heresy. He died about 175 A. D. Of his many works there came down to us but one oration, which is lengthy, and it is against the Hellenes or pagans, in defense of christianity.

What is especially important, Tacian compiled a summary on the ground of the four gospels, and which he briefly named \textit{Of the Four} (Diatessaron). As this gospel was compiled literally according to the four gospels accepted by the church, it was in use among a considerable portion of orthodox christians in the east for a long time for its briefness, and also for less difficult labor in copying, it was preferred to the complete four gospels. Theodoretos, the Bishop of Cyra, during the first part of he fifth century found more than 200 copies of this gospel in his diocese. He found nothing in them which did not agree with the universally accepted gospels, only that Tacian has omitted the geneology of Jesus Christ and the account of his birth according to the flesh from the seed of David. In its place Theodoretos distributed the original four gospels, and thereby weakened the memory of Tacian's sad case of heresy.

From all the above mentioned it is clear that St. Justin, the teacher of Tacian, had accepted the original four gospels, no more and no less, as the church at first authorized, including of course the gospel of St. John. This assertion is strengthened by the canon of New Testament books, which has come down to us, and which has been in use in the church of Rome during the time of St. Justin. This canon was discovered and published by the learned Muratori. The first part and the end have been lost; it commences thus: ''the third book is the gospel according to Lukke.'' Briefly commenting upon this gospel, the writer of the canon continues: ''the fourth of the gospels is John's, one of the disciples,'' and briefly telling about the writing of this gospel by John the Divine, he goes on, making mention of the book of the ''Acts of the Apostles,'' written by the holy evangelist Luke. There is no doubt that the first two gospels in the Roman church were those of Matthew and Mark, which were known at the close of the first century to the apostolic disciple, Papius, the Bishop of Hieropolis. The Gospel of Mark itself was written by him on the request of the Roman christians; for this we have the testimony of St. Clement of Alexandria, a learned man of the second century; a proof of this is also the gospel iteself, which contains a considerable number of Roman words.

Soon after the death of Pius the first, during the time of his successor, Bishop Anecetas, about 160 A.D., as we have mentioned before, the aged Polycarp, the Bishop of Smyrna, a disciple of St. John and other apostles, came to Rome on account of a dispute concerning the time of celebrating the day of the resurrection of our Lord, which had arisen between the churches. And, of course, it was necessary to turn to the gospels while speaking about this subject. If the Gospel of St. John, undoubtedly helf by the Roman church at the time, was not authentic, St. Polycarp would surely bring the case forward, and a dispute would have arisen concerning this Gospel, or the Roman church would have excluded it from the list of New Testament books, but there was no such proceeding; consequently, the gospel of John belonged to the apostle himself.

It must be taken into consideration that in all the works of the apostolical men and other writers of the first and the earlier part of the second century there is expressed a clear faith in Jesus Christ, as the Son of God, who was incarnate of the Most Holy Virgin on earth, who worked great signs and miracles, who arose from the dead and ascended into heaven, and it is also clearly shown that the christians of this period had known none other Christ, but Him who is revealed in the Gospel.

No less important for the proof of our subject, at least for these careful treasurers of the truth in the eaarly centuries, is the example of holy life and also the tradition o fhte very first christians, of whom there were more than 500 that had seen Jesus Christ. It was but the tenth day after our Lord had ascended into heaven, and when He sent down upon the church power from the Almighty, that over three thousand more were added to the followers of Him, whom great multitudes of people on many occasions followed when he walked upon earth with his twelve, who were themselves in their first simplicity rather sceptical believers of only the real--which could be seen with they eyes and felt with the hands.

I hope you will bear cheerfully a moment or two longer, while we bring these testimonies of ancient literature to a close. Yet, I must admit, it has been for your sakes, that this subject is considerably abridged.

About 130 A.D. the holy martyr Codratus, who was Bishop in Magnesia, was put to death for his faith in Christ. He was a disciple of the apostles and a prophet. The reverend-martyr Codratus is known for his writing which he handed to the Roman Emperor Adrian in 126, in defense of the persecuted christians. Although this apology has not reached our times, yet it was read by Eusebius the historian, who quotes from it the following: ''The works of our Savior (writes Codratus to the Emperor) have always been manifest, because they were trurhful. Those whom He healed and resurrected from the dead were visible--not only when He healed and resurrected them, but always. They lived not only during His life on earth, but remained considerably longer after he left us; some of them have lived to our day,'' i.e. to the time between 80-90 A.D., when Codratus was in the prime of life.

In support of this investigation we bring forward historical facts, and when such come from heretics or even unbelievers, they are the more valuable, as the work is shown thereby to be impartial, and the Divine to be above the need of a human justification.

Thus, another heretic, Valentine by name, boasted that he had received his teaching from Theodala, a disciple of the apostle Paul. Valentine preached in Egypt between 120-130. He settled in Rome about 140 A.D. According to the testimony of Irenaeus and Tertullianus, writers of the second century, Valentine had made use of the whole of Sacred Scriptures. In the book of St. Irenaeus against heresy, not only many parts of the gospels are mentioned, but also quotations from the holy apostle Paul, which were perverted by the unorthodox explanations of the Valentinians. The reverend-martyr Ippolitus ascribes to Valentine these words: ''all the prophets and the law speak through Demiurgus, the god of non-reason.'' Therefor the Savior saith: \scripture{all who came before me, the same were thieves and outlaws.} Now this text is found only in the gospel of St. John. It is known also that Heracleon, a disciple of Valentine, had written a commentary on the gospel of John, considerable parts of which have been preserved in Origen's commentary on the gospels. Heretics as well as others, who lived in the second century, would not have accepted of the church the gospel, moreover a false one, if their founders and teachers who lived in the time of the apostolic men, were not firmly convinced of its authenticity. St. Irenaeus in the second half of the second century wrote thus: \scripture{Our gospels are strongly established,} even heretics become their witness; quoting from which they expect to uphold their doctrine. Heresy, which began to show itself so early among christians, was the cause nevertheless why the first bishops and other leaders of the churches kept so diligent a watch over the gospels in their original completeness, and guarded them from any and all mutilation, which some of the heretics attempted, but of which they were accused in good season.

Only recently, in a responsible magazine published in the eastern metropolis of America, October 28, 1899, the profoundly learned Professor William C. Winslow, D. D., D. C. L., L. L. D. writes: ''Among the papyri discovered at Behnesaa by the Egypt Exploration Fund is a fragment of the Gospel of St. John, which proves to be of the highest importance and deepest interest. It antedates all our previously known texts by one hundred years or more. Our associates have now completed their critical study of the text, and a facsimile of it will appear in our volume about ready for the press, with a great many docuents of the first century translated. The papyrus of the first chapter of St. Matthew (A. D. 150), corroborating our version of St. Matthew 1: 18-21, and the Logia (New Sayings of Christ) were in book form. This fragment of St. John is also in book form.

It has been assumed that the form of writing in a book or codex dated from the introduction of vellum; but the foregoing and like discoveries by the Fund show that such fashion was in use for christian literature of the earliest times.

The St. Matthew and Logia fragments are in single leaves, but the papyrus of St. John is on a sheet, and is written upon both sides. Moreover, the first leaf contains St. John I., and the second leaf St. John XX. in part; so that we possess one of the outer sheets of a large quire between which and chapter XX. were the intervening eighteen chapters, now lost. This book of the Gospel contained about fifty pages.

It is to be noted that the usual contractions for theological words like God, Jesus, Christ, and Spirit are used. If such contractions were familiar in the second century, they must have been introduced much earlier. Do they not show the existence of a christian literature as early as 100 A. D.?

The text, a small uncial, resembles that of the Codex Sinaiticus, to which variants of its own are added. But the facsimile will reveal this and other characteristics to the scholar who sees our coming volume, and the christian public will be deeply interested in the publication of a text containing the statement, '\textit{The Word Was Made Flesh,}' which words were accepted in the early morning of christianity as very truth. \textit{Out of have I called my Son} may be transliterated for today: 'Out of Egypt comes the proofs for the Bible as God's revelation to man.'''

Christianity is not so much in danger of a so-called ''learned unbelief,'' as it is of the ''little faith'' and practical unbelief in active life, and the spreading of unchristian habits and customs, and in danger from the life even of christians, which is not according to the gospel. Many, Oh many so-called christians of our times are in need of having their hearts renewed, and deeply impressed with the words of the Savior ''\textit{repent and believe in the gospel.}''

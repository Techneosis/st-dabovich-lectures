\chapter{The Immortality of the Soul}

There are many proofs testifying to the immortality of the soul, and they differ by degrees of strength, and the source from whence they proceed. There are proofs of supposition, then  positive ones, and, finally, decisive proofs. We have some in visible nature, others are had from the attributes of God's being, and from the nature of our soul, and then there are open and experimental proofs.

What do we see in nature? A change of one and the same being from one condition to another higher form of existence. This is observed more clearly, especially, in the world of insects. Take, for an instance, the creeping worm and the flying butterfly, etc. In the same manner does man here on earth undergo two principal forms of life, to say nothing of his ages.

First he is formed and lives unconsciously in his mother's womb, then, on being born into the world, he lives on the earth. This life is incomparatively higher than the first, yet here the days for man are exchanged by dark nights; after days of good fortune there come days of misfortune. It is natural for him to await a third and better life, when he will freely move in the space over the earth, where it is eternal day, and where there is no sorrow nor sighing.

Of course this proof is suppositional, but it contains no small degree of convincing power. The Lord himself has implanted in our nature types and foretellings of the better life which is to come.

We also see that the animal kingdom, and man as well, are gifted with a generic immortality. By transmitting life from generation to generation, they will last as long as the earth continues in its present condition. And why cannot God give to the highest of all creatures, and to him who rules over all animals -- i.e., to man-- besides a generical also a personal immortality? The Lord is all-powerful. He can create everlasting beings. He is all-good, and the best good is a life which has no end and knows no grief.

The highest being on earth,--i.e., man,--were he not destined everlastingly to glorify the eternal God, and were he not immortal, the earth would be a sorrowful spectacle of the all-destroying and all-devouring death, and God would be only the God of the dead, and not of the living upon earth.

There are yet other proofs of the immortality of the soul, which rest upon two principles combined: First, the existence and attributes of God; second, upon the nature of the soul.

The first positive and clear proof of this kind is the following: Without a doubt there is a God; because He reveals Himself to man, not only in his soul, but in visible nature, and by a direct manifestation of Himself. He is a being all-just, all-holy. In the nature of man's soul he has implanted the aspiration for good deeds, and the aversion for evil; and there are many people who try to do right and accomplish holy deeds, while at the same time they bear heavy trials of self-renouncement. Yet we see on earth that sinners often enjoy good fortune and success, and the just suffer till death, and mostly at the hands of sinners.

If there was no other life for people, in which there must be a reward according to actions, then God would not be all-just and holy. He would not be merciful to the good, while showing mercy to the unjust ones, and this cannot be conceived of God. Therefore, as God is holy, there will be another life, in which the sinners and the righteous will receive their just reward.

Another proof of the immortality of the soul is like this, but wider in its contents. In the mind of man God has implanted a desire for the truth, in his will a striving for the good, in his heart an inspiration for happiness; but the mind of man does not become satisfied with the knowledge obtained on earth. He sees that it is not, by far, complete and perfect. His will meets with much resistance in growing in the good. Although man labors much for the good while on earth, yet he finds himself, by far, undeveloped in a moral sense; he feels the burden of sin; his heart finds no true blessedness on earth. The thought of eternity and the everlasting is rooted in our soul. It is not only that the soul possesses the idea of the infinite and most high Being, but with the heart and will itself it yearns to approach Him.

Furthermore, what meaning is there in the fact that people, by different ways, endeavor to perpetuate their name upon the earth, desiring to be remembered as long as possible after their death?

All mankind, with the exception of a few individuals, believe in the future life. Why should the Creator implant in the spirit of man such lofty aspirations, if they were not to be realized, if such hopes were not fulfilled and desires not satisfied,--i.e., if there were no better, everlasting existence? It would not be according to the goodness, and the wisdom, and the holiness of the Almighty, of which we know.

Therefore, as it is beyond doubt that there is an eternal and all-perfect God, it is also beyond doubt that there will be a life without end for man, in which the longings of his spirit, or soul, will be satisfied.

Also, many uncommon manifestations of the powers of the soul prove its non-materialistic and everlasting qualities. Such are: Predictions of events in the future; visions of what is going on in another part of the earth, at a great distance; and distinct foresight of the future, sometimes at a very long period in advance, and most often in images, not only in sleep, but also in a wakeful condition.

The most definite and decisive proofs of the immortality of the soul are the theological ones which we have from Divine revelation, and also experimental ones, in the appearance of the souls of people who are dead.

God has many times revealed Himself to people, in different ways, and he still does so, through His saints, chiefly in prophecies, signs, and miracles. But why has He revealed Himself to people, and does so still He has and yet does so in order to prepare us for the next blessed life.

God has not only revealed Himself to mankind, but he has shown us the way and the means by which to attain the better, heavenly life.

The Lord, in his revelation, says that the soul will live eternally, that it will never die, and that man, in his very body, after the resurrection of the dead, cannot die. (Luke xx. 36.)

It would fill a large book to relate all the appearances of the soul known of in the ancient world. And since the time of Christianity you have read or heard of the miracles in the lives of the saints.

If we were to gather authentic facts pertaining to visions from the spirit world in a period nearer to our day, why, they would make up several volumes. Not only saints have appeared, but sinners, and the souls of common people were revealed on earth for the knowledge and the assuring of the living in the existence of a future life.

Although it is forbidden Christians to thrust themselves, of their own accord, reaching out for still more evidence, into the spirit world, as well as it is strictly condemned by the Church systematically to practice the occult sciences, yet, on an occasion like this one, we may, for your proper information, relate, of the many, at least a few such instances.

When the Queen of Sweden, Ulrica, had died in the castle of Gripsholm, her body was laid in a coffin, and in the front room a company of the Royal Guards were on duty. Punctually at noon-time there appeared in the parlor the Countess Stenbok, from the capital of Stockholm, whom the Queen had loved, and the commander of the Guards led her to the body of the Queen, and left her there.

As she did not soon return, the Captain opened the door, but struck with horror, he fell back. On running to his aid, the officers then present saw, through the open door, the Queen, who was standing in her coffin, and embracing the Countess Steinbok. The vision seemed to float, and then changed into a heavy mist. When it soon cleared off, the body of the Queen was lying in the coffin, and the Countess Steinbok was nowhere to be found in the castle.

At once a messenger was dispatched with information of this incident to Stockholm, but word came back, that the Countess had not left the capital, and that she died at the time when they saw her in the embrace of the Queen.

Immediately a statement of the facts was written down, and signed by all who saw this vision.\footnote{TODO}

Much has been written of the appearances of the \textit{white woman}, at different times, in the palaces and castles of Germany, foretelling the death of a member of the royal family. Judging from a portrait which was found, she proved to be the Princess Bertha von Rozenberg, who lived in the fifteenth century.

In December, 1628, she went through the apartments and halls of the Berlin mansion, and distinctly pronounced these words: ''Come, thou Judge of the living and the dead. There is yet hope of a judgement for me.'' During the present century she appeared to the Queen Louise twice, and later on,--on other occasions, which you may remember, having read of it in the newspapers.

Leaving these narratives, of which there are many in the histories of all nations, we shall now take up some facts, which are more appealing and pleasant, for the reason that they concern persons who are especially dear to us, being members of the Orthodox Church of Christ.

In 1855 there lived in the town of Epiphany, in the Russian province of Tula, a merchant, by name Basil Ivanov R., who held the office of church-warden at the cathedral. His family consisted of himself, his wife, two sons,--Nicholas, 23 years of age, and married; John, 17 years old,--and three daughters,--Catherine, 13 years; Raisa, 11; and Alexandra, 6 years.

On the 10th of April, 1855, which was on the eve of the Sunday of the Holy Myrrh-bearing Women, Raisa, from fright, was suddenly taken with strong convulsions, while at the same time she evoked curses against God and His saints, but especially against St. Sergius. Medical aid afforded no relief to the unfortunate child.

Two months after the disease began, on the 11th and 12th of June, Raisa was taken with still more terrible convulsions. Becoming conscious, she said that St. Sergius appeared to her, the he talked with her, and brought her a church loaf, which she ate. It seemed as though she ate, her relations said, but they could not see the \textit{prosphora}, and attributed this to her sickness. The unbelief of her oldest brother, Nicholas, was especially painful to the girl.

In response to the instructions of St. Sergius, she requested all in the house to repeat the prayer, \textit{May God arise}, and to make the sign of the cross. With great difficulty the mother set her fingers and made the sign of the cross upon her. It appeared to her that evil spirits were leaving her, as she saw, on making the form of the cross.

By another intimation from St. Sergius, an image of himself was found in the corner behind the sacred pictures, and this was put upon her.

Desiring to convince her brother of the fact of the appearance of the holy man with a church load, she asked for a glass of clean water, and taking some of the water into her mouth, she let it out again. Then could be seen crumbs of white bread in the water, which no one had given her.

After this she announced that the saint would come to her on the 13th date. At eleven o'clock in the evening of the appointed day, another spasm most frightfully shook her. In half an hour after, she arose and said, ''Here comes St. Sergi,'' and then went to the window. Having opened the window, she let out her arms and began talking to some one. After this she turned to her brother, and giving him something in a paper, tells him to hold it with reverence, as something holy. Then was given her, from below the window, a cross made of white ribbons. After showing it, she said that she was told to return it. At this her fce became bright. Her parents and the home folk at the same time felt a happiness and reverent fear.

Notwithstanding all their requests to give them the cross, she let it down in her hand out of th window, and it disappeared. The family ran out of the house, which was a one-story building, behan to look for the cross under the window, but could not find it.


After this she requested her brother to show what he held in his hand. There was found in the paper a corner-formed particle of a church loaf, and some pieces of incense. Upon the paper these words were written: ''It is Thou, O Lord.''

On being questioned by her relations, she said that St. Sergius came to her in company with a beautiful lad who was girdled with a deacon's stole. The holy man took out from under his cloak a napkin, out of which he gave her the paper and the cross. The paper with the prosphora nd incense he ordered to be kept, but the cross to be returned. He also gave instructions that the girl's relations should believe, especially the brother Nicholas. Concerning the lad who appeared with him, he said: ''He will guard you.''

After this the sick girl became entirely well. The next year, in June, 1856, she traveled with her mother and sister to the Troitsa Monastery, or Lavra of St. Sergi, near Moscow, and her mother told the superiors of all that hapened. The particle of church bread, which she brought with her, proved to be the baking of the Monastery.

These events were written down, and as facts they were attested to by the signatures of the girl's father, mother, and brother, and then by the archpriest, the priest, and the deacon of the St. Nicholas Cathedral in Epiphany, by each separately. Still this occurence was not published in print for six years, only after six years, when it was learned that Raisa still continued in good health, it was published with the consent of Metropolitan Philaret in the journal of the Theological Academy of Moscow, This appearance of St. Sergius occurred 464 years after his death.

In 1812, when Napoleon entered Moscow, Prince Eugene, the Vicerow of Italy, with a vidision of warriors left Moscow for Zveniqorod to pursue the Russian partisans. The Prince occupied rooms in the Monastery of St. Savva, who was a pupil of St. Sergius of Radonej. About ten o'clock the Prince, without undressing, lay down and fell asleep. In the meantime he sees a man in a long black habit--whether asleep now or awake, he did not know; by the light of the moon he could see the man walk close up to him; he was old, with a grey beard. Then the visitor said to him: ''Give thy men orders not to plunder the monastery; especially see that they take nothing from out the church. If thou compliest with my wish, then God will have mercy upon thee, and thou shalt return to thy fatherland well and in safety.''

Nextt morning the prince gave command that the division should return to Moscow; first, on going into the church, he saw by the tomb of St. Sabbas a picture of the man who appeared to him, and recognizing the image, he reverently knelt before the relics of the saint and then took down in his diary a note of all that happened.

All the marshals of Napoleon came to their end unfortunately, but Eugene remained safe and was nowhere wounded in a battle after this. He expressed his will to his son Maximilian that, should he ever visit Russia, to go and offer his veneration at the tomb of St. Sabbas. The son came to Russia in 1839, during the reign of Emperor Nicholai Pavlovich, and after the military maneuers on the field of Borodina, in memory of the battle of 1812, he inquired of the whereabouts of the monastery of St. Savva, went there with guides, and knelt before the grave of the holy man. This event is so well known in history that there can be no doubt.

And so, this earth is a place for our temporal residence. It is the nurdsery of reasonably beings for other worlds. Thus, we will, and may be soon, taken from the earth. Let us pray the Lord, that He give us strength to leave our sins, which bring us down into the dark spaces of the universe, and that He help us to accomplish deeds which take us up into the high, eternally bright mansions of heaven. Amen.

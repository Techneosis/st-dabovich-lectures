\chapter{Preface}

In this book I offer to the English-speaking public in general, and to the American in particular, a historic, theological, and moral review of the Orthodox Eastern Apostolic Church, commonly called the Greek-Russian Church, in the form of lectures and sermons, thus enabling them to see the actual practice and teaching of a Church which is making herself at home in the West, notwithstanding her birth in the East, and which knows none other head but Jesus Christ.

Now and then people are told, chiefly through small journalism, that the Emperor of Russia is the head of the Orthodox Church. There are some who accept this view, and these belong to two classes: the uninformed and the prejudiced. It must be made clear that the Orthodox Church has three of her Patriarchs residing in the Turkish Empire, while about 6,000,000 of her members are the subjects of the Emperor of Austria, besides which we have to count the Kingdom of Greece, the Church in Egypt, and three independent branches of the Church in the Balkan States. We simply mention the purely characteristic Orthodox Church in Japan, the missions in China, America, and elsewhere, together with the Church's congregations in all the countries of Europe, which though peculiarly original, in regard to their local premises, are, nevertheless, in spiritual relationship with the Great Church of Holy Russia, relying upon her for the larger portion of their support. And the Russian Church, enjoying in complete measure the sympathy of the Orthodox and highly pious Ruler of the Empire, does not begrudge Orthodox missions her support, which is so often made the subject for taunting with suspicions by outsiders who are strangers to the Christian spirit of toleration. Thus it is clear that the Orthodox Church has no Pope-head; she is not a monarchy, but as the Church of Christ she is Catholic and Apostolic. Indeed, it would seem strange to say that in this country Queen Victoria is the head of the Episcopal Church, because the Anglican and Episcopal Churches are in close communion.

Readily do we acknowledge, and sometimes too hastily adopt, the results of the great achievements of the Western mind and spirit in the affairs of this world; but in matters of faith the Eastern is, as it has ever been, the source and cradle of everything that is purest, highest, and heavenly. Humiliating though it might appear to the haughty spirit of the West, it will at last, and of necessity, turn its eyes towards the East and realize the saying: \textit{Ex Oriente lux!}

Sebastian Dabovich,

A priest of the Holy Orthodox Church.

San Francisco, on the day of our Lady of Kazan, 1899.

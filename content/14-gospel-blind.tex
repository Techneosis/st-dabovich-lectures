\chapter{Sermon for the Sunday when the Gospel of the Blind is Read.}

(Read first the ninth chapter of St. John.)

THIS is the Gospel for to-day. What lesson 
have we to learn on this day? We must 
find the substance in these words, and feed on it, 
for it is spiritual food. When we have digested 
this Divine food, it will be assimilated with our 
natures, and our humanity will become purer, 
brighter, stronger, yea — and perpetual, so long 
as it lives with the Word of God, for hath not 
the Savior Himself said when the devil tempted 
Him who hungered in the wilderness: That 
man shall not live by bread alone, but by every 
ivord that proceedeth out of the mouth of God? 
So then, have we considered the Gospel while 
being read? If so, we find that the principal 
subject of it is the miracle which was worked by 
our Lord Jesus Christ. Next, we observe the 
man who was the object of the miracle, and 
finally we get a perspective of the condition. 
The circumstances which surrounded this miracle
were most unfavorable for the blind man's 
confirmation in the faith, although he succeeded 
against such materialistic odds, and likewise for 
an open manifestation of the glory of the "Wonder-worker
Himself, yet the greatness of which 
became the more conspicuous as passion-bound 
opinions, systems and classes strived to overcome
or, in the least, to belittle it. 

When 1 stop to meditate, it seems that I am 
transported to the green hills of Judea, where 
the common folk of both hill country and populous
valley are all astir with lively discussions 
in the midst of their every-day duties, as in their 
homes they go about to and fro, and, mind you, 
it is all about religion and politics; religion first 
and politics after — insomuch as it is related with 
the proud nature of a people, who boasted of 
being the chosen race of God, who expected His 
messenger, and were to be ruled by none other 
than the Messiah Himself, unto all ages. It 
was a day of expectations, indeed. The intellect 
of the masses had been sharpened to a turning 
point. The very "times" themselves were full 
with signs. Everybody was inquiring. The 
people willingly divided themselves into two 
sets : those that taught and those that were 
taught. The nearer that some of them had gotten
to the truth, the more danger there was of
taking falsehood for the truth, and thereby more 
danger of two blind men falling into one pit. 
Passions, although with a semblance of a higher 
quality, yet human and materialistic, ruled the 
hour. In such a midst Christ, the only true 
teacher of men, had come. No one condemned 
false doctrine so energetically as this teacher 
had done, and no one had taught with such invincible
strength and power as He did. Now 
the whole company of teachers arose against this 
One, and, notwithstanding their divisions, they 
knew how to agree in one and the same decision 
which suited them all, and that was :, Build That He led 
the multitude astray (John vii: 12), He speaketh
blasphemies (Luke v: 21), He perverteth 
our nation (Luke xxiii: 2), and, at the end, for 
His teaching said they : He is worthy of death 
(Matt, xxvi: 66). But they could not destroy 
the work of Him, whom they hated, for the people
did see in Him The Great Prophet (Luke 
vii: 16). Above His calling as a teacher, He 
had the merits of a miracle worker. What now 
could His angry enemies do or say against this? 
Now they would do as they have done at that 
time, viz: murdered Him. But His works remain,
and for that the glory of His resurrection 
is the brighter. When the different conditions 
of a changing world, together with the many 
representatives of opinions have exhausted their 
machinery, all their means, and wasted their fine 
scholastical dialectics, while the simple facts, 
told by him who had once been blind, remain as 
simple facts, which he — who now sees — will not 
renounce, then society answers and says to the 
followers of Jesus Christ: "You were altogether 
born in sins, and do you teach us?" When Christians
cannot be subdued, nor compelled to follow 
the ways of politicians or the world in general, 
then they are left all to themselves. And they 
cast Him out. 

The Son of God, manifesting His power in 
miracles that we may desire Him alone and 
thereby become strong in faith — this is the lesson
that we are to learn to-day. Now the learned 
tell us that the nineteenth century (which happily
is in its death-throes) requires "advanced 
thought." I wish the nineteenth century was 
over; we have heard it bragged about so much 
that one actually gets sick with the nineteenth 
century. We are told that this is too sensible a 
century to need or accept the same Gospel as the 
first, second and third centuries. Yet these were 
the centuries of martyrs and confessors, the centuries
of heroes, the centuries that conquered all
the gods of Greece and Eome, the centuries of 
holy glory, and all this because they were the 
centuries of the Gospel. But now we are so 
enlightened that our ears, strange to say, really 
ache for something fresh, and under the influence
of so-called refined literature (how about 
ordinary novels?) our beliefs are dwindling 
down from mountains to ant-hills, and we ourselves
from giants to pygmies. 

By God's grace some of us abide by the 
Orthodox Faith, and mean to preach the same 
Gospel which the saints received at first. It is 
a foundation which we dare not change. It must 
be the same, world without end, for Jesus Christ 
is the same yesterday, to-day, and forever. Amen.

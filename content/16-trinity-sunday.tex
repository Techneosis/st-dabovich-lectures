\chapter{Sermon Preached on Trinity Sunday.}

There are three that bear record in heaven, the Father, 
the Word, and the Holy Ghost; and these three are one. 
(John v:7). 

THE Orthodox Faith is this, that we worship 
One God in Trinity, and the Trinity in 
Unity, neither confounding the persons (hypostasis),
nor dividing the substance (Symbol 
d'Athanasius). What will it profit us when we 
study deeply concerning the Trinity, if we be 
found lacking in humility and thereby are displeasing
to the Most Holy Trinity? It is not 
for us to search into the incomprehensible mystery
of Divinity. If we would approach any 
where within a reasonable approach to the Divine 
Mind, we, who are of the lower ones, should 
always endeavor to be pleasing companions to 
those who are of the higher ones, so that we may 
fervently glorify the thrice illumined Deity together
with the angels, singing with that faith, 
which is the assurance of things hoped for, the 
e st of things not seen: Holy, Holy, Holy, Lord
of Sabbaoth, heaven and earth are full of Thy 
glory. 

"Some may say," as St. Cyril of Jerusalem 
lias long ago rightly surmised, " if the nature or 
substance of God is infinite, then to what purpose
do we speak of it? But shall I abstain 
from taking water out of the river for my use, 
though a small measure, because I cannot drink 
up the whole river? Is it because my eyes cannot
contain the whole of the sun himself, I should 
not, as much as is necessary, make use of the 
daylight? And if I were to enter some large 
garden, the fruits of which I could not eat up, 
would you, for this reason, have me leave the 
garden altogether hungry?" Indeed we need 
the water, we need the sunlight, and the fruits 
of good endeavor and religious labor we enjoy. 
Therefore we should, and it is our bounden duty 
to learn of that knowledge, which the Creator 
has been pleased to reveal of Himself to His 
Church, without giving ourselves to vain speculations,
and probing into mysteries not necessary
for our temporal welfare, much less so for 
our eternal salvation; and while not attainable 
to our limited mind, sometimes because of the 
presence of sinful pride, sometimes on account 
of deceit and feeble support in false systems
and individual schemes, but chiefly because a 
drop does not contain the ocean, a particle of 
creation cannot embrace the earth, the worlds, 
and that wisdom and power which ordered the 
universe and established the laws by which it is 
preserved. ' 

To learn of the Supreme Being and concerning
the Holy Trinity in the God-head, we need 
not go far; the philosopher cannot make clearer 
the light itself; we need not question the astronomer;
and surely the majority of mankind only 
become puzzled when they singly weigh the corroborations
of the geologist. Just look around 
and you will see on the space of a few yards of 
earth a great quantity of heterogeneous beings 
and natures, existing in the same air and by the 
same material food in substance as we do; yet 
some are adapted to life in the air, some move 
only in the water, and others are subjected to 
several certain limited forms of existence upon 
the land, while man, by his intellect and by his 
will, adapts himself to his surroundings in such 
a measure that he controls all other forms of 
animal life, and even overcomes natural obstacles 
by a force which is above nature. Thus we see 
man is created in the likeness of the Most High 
Creator. 

After all these ages we cannot find in the advanced
theories of modern literature an hypothesis
that may compare with the plain statement 
of facts by the Holy Fathers during the first 
centuries of Christianity, as we have them concisely
epitomized by St. John of Damascus. He 
says: "The Divinity is indescribable and incomprehensible.
For no one knoweth the Son, 
but the Father, and the Father no one knoweth 
save the Son (Matt, xi: 27). And also the Holy 
Ghost knoweth that which is of God, as the 
spirit of man knoweth the things of a man 
(1 Cor. ii: 11). Beside the first and blessed 
Being no one ever knew God, unless God had 
revealed Himself to some one ; no one, not only 
of mankind, but no one of the celestial powers, 
nor of the Cherubim and Seraphim. Yet God 
has not left us entirely ignorant of Himself. 
The very knowledge of the existence of God 
the Creator has Himself implanted in our nature. 
And creation itself, the government of nature 
and its preservation proclaim the greatness of 
God. Above this, first through the law and the 
Prophets, then through His only begotten Son, 
our Lord and Savior Jesus Christ, God revealed 
to us as much knowledge of Himself as we are 
capable of containing. Therefore, all that has
been given us by the law, the Prophets, the 
Apostles, and the Evangelists, we accept, acknowledge,
and respect, and we seek nothing more. 
Thus God, as the Omniscient One and the Provider
of that which is profitable for each one of 
us, has revealed all that is for our good, and kept 
in silence that which we are not capable of containing.
Being satisfied thereby, we will keep 
to this, not transfixing the borders of eternity 
and not overstepping Divine tradition." 

In the light of Divine revelation it becomes 
clear to our reason that, excepting the one, true, 
most perfect God, another cannot exist, because 
the most wise, most powerful, the most High and 
perfect Being must be only one, beside whom 
there is no other. 

The Christian Faith is the religion of the 
Most Holy Trinity. And this is nowhere so 
plainly demonstrated as it is in the books of the 
New Testament. The preacher with the hearers 
and doers of the Word would be unequal to the 
task to take up at this moment for examination 
the great number of testimonies we have in the 
history of the New Testament concerning the 
Orthodox Faith, in which we worship One God 
in Trinity and the Trinity in Unity. But this is 
not all. If you turn to the books of the Old
Testament, it can be seen that the Patriarchs of 
the nations and the saints of old had almost as 
clear a conception of the nature of God as we 
Christians, and believing in One God they at the 
same time worshipped, more or less consciously, 
the Father, the Son, and the Holy Ghost. The 
Most Holy Trinity, while distinct in Persons, is 
of equal Divine Substance, and equal Majesty. 
We do not belittle the awful magnitude of this 
Truth when we follow the example of the Holy 
Fathers by taking various illustrations from 
created life to help us in some measure grasp 
this doctrine of the Divine Three in One, and 
One in Three. In man's soul the image of God 
is more or less reflected. Take for instance the 
memory which recalls the hour of Divine Liturgy,
the understanding which reasons upon the 
duty of public prayer, or considers the excuses 
that might be pleaded for staying at home, and 
then the will which chooses one course or the 
other. 

And now let us remember that, although we 
are among the weakest members of all creation, 
we may become, by being faithful Christians, 
partakers of the Divine Nature (2 Peter i: 4) 
that by the grace of the Holy Ghost, who has 
regenerated us, we are all the children of the
Heavenly Father and brethren to His only begotten
Son (John i: 12, 13, Luke xiii: 21). Thus 
we may come into the closest moral relations 
with the Triune God Himself, if only, believing 
in Him and earnestly drawing toward Him with 
hope, we shall love Him with our whole heart, 
with our whole soul, and with our whole mind 
(Matt, xxii: 27). Then, without doubt, the promise 
of the Savior will be fulfilled for us. Jesus said: 
If a man love me, he will keep my word, and 
my Fedher will love him, and will come unto 
him, and make our abode with him (John xiv: 13). 
And then also we will understand the meaning 
of these words of the Apostle: Know ye not 
that ye are a temple of God, and that the Spirit 
of God dwelleth in you? (1 Cor. iii: 16). Amen.

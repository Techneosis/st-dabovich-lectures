\chapter{Sermon for the Fourth Sunday in Great Lent.}

For He {Jesus) taught His disciples, and said unto 
them: the Son of Man is delivered up into the hands of 
men, and they shall kill Him; and when He is killed, 
after three days He shall rise again. (Mark ix: 31). 

CHRIST the Saviour, having spoken the sorrowful
word that they shall kill Him, adds 
the joyful ones that He on the third day shall 
rise again, concluding thus that we may know 
that after sorrows there always follows happiness.
If there were no temptations there would 
be no crown, no hardships, no rewards; were 
there no conflicts, nor would there be any honors,
no sorrows, no comforts; if there were no 
winter there would be no summer. 

And this we may observe not only in people, 
yet also in the seeds which are thrown into the 
earth; and here a heavy rain and much cold are 
necessary, so that a stem spring up green, bearing
its ear of plenty. Let us sow also in the 
time when spiritual misfortune visits us that we 
may reap in the summer; let us sow tears that we may reap happiness. According to the 
Prophet of God: They that sow in tears shall 
reap in joy (Ps. cxxiii: 5). Not so beneficial 
is the rain which pours over the seeds as the 
rain of tears, which gives the power of growth 
and ripens the seed of piety. As the tiller of 
the soil cuts deep into the earth with his plow, 
preparing a safe place for the seed that they 
may hide in the very bowels of the earth and 
safely take root, thus also should we with misfortune
and sadness, as with a plow, touch the 
depths of our heart. 

The holy Prophet would convince us thus, 
saying: Tear open your hearts but not your 
garments. Let U3 tear open our hearts so that, 
if there be any evil plant or evil thought within 
us, we may pluck it out with the root and cleanse 
the field for seeds ct holy devotion. Lf we do 
not renew the field now, if we do not sow now, 
if we do not shed tears now, when it is Lent — 
in this time of sorrow and fast — at what other 
time, then, shall we be afllicted ? Can it be in 
the time of ease and pleasure? No, it cannot 
be then, for ease and pleasure lead to carelessness,
while sorrow compels the soul, which is 
beset with many attractions on all sides, to look 
within itself. 

The farmer having sown the seeds, which he gathered with much labor, prays for rain; and 
one, not knowing the work, with amazement 
looks upon all, and, perhaps, thinks so within 
himself : "What is that man doing? He throws 
away that which he gathered; but not that only, 
he yet carefully mixes it with the earth; and 
that is not all, for he prays that what he has 
sown may decay." Quite contrarily does the 
farmer when he sees the coming clouds overshadow
the sky he rejoices, for he does not look 
at the present, but to the future; he doe3 not 
tliiuk of the thunder, but of the sheaves; not of 
the decaying seels, but of the yellow ripe stalks. 
Thus should we look, not at the sorrow of the 
present, but at the benefit which is derived from 
it. If we be on our guard, we will not only 
suffer no evil from sorrow, but derive much consolation;
but if we be careless, the very enjoyment
of quiet will turn to be hurtful for us. To 
the careless one thing and another is evil, but 
to the diligent one thing and another is profitable.
As gold retains its brightness when it lies 
in the water, and becomes still brighter when it 
is cast into the furnace, so we see the very opposite
when if clay and straw are put into water; 
the one dissolves, the other rots. Now this is 
just the case with the righteous and the sinner; 
the first living in quiet remains bright like the 
gold which was put into the water, and being 
afflicted with temptation becomes brighter still, 
as the gold which passed through the fires of 
the refinery. But the sinner, although enjoying 
ease, dissolves and decays, as the straw and clay 
thrown into water or the furnace, where it burns 
and perishes. 

Let us not be sad for present misfortune. If 
you have any sins, they will be easily uprooted 
by sorrow. And if you are the possessor of a 
virtue, it will become brighter for having undergone
temptation. If you will continually watch, 
you will be beyond the reach of harm, as the 
cause of falling generally is not the kind of 
a temptation, but the carelessness of those 
tempted. And so, if you would enjoy quiet, do 
not seek pleasure, but strive to make your soul 
capable of being patient, because if this quality 
is wanting in you, you will not only be conquered
by temptation, but sooner fall a prey to 
the spirit of desolation which ease will bring 
upon you. As the storms of wind do not uproot 
a strong wood with its roots, but on the contrary 
from constant blowing on from all sides it becomes
firmer, so does the holy soul, although 
overwhelmed with afflictions, it bends not, but 
becomes invigorated with a higher energy. How 
might we, a generation of the New Covenant,
become justified and forgiven when we with 
difficulty overcome human temptations, while 
Job, the much afflicted, outbore with such alacrity
a most sore temptation in the days before 
Grace, of the Old Dispensation? 

Are you sad, beloved, because the Most Good 
Provider through sorrowing brought you to the 
thought of eternal salvation? God can put an 
end to-day to all troubles; but He will not destroy
sorrow until He sees a change in us, until 
repentance really and strongly works within us. 
The goldsmith will not take the metal from the 
fire before it is purified, and God will not withdraw
the tempest clouds before we are perfected 
by corrections. He who sends the affliction 
knows when the time comes to hold its stay. 
A player of the cithern does not tighten too 
much the strings, else they snap, nor does he 
freely loosen them, else the harmony of sound 
be lost. So does the Lord with our souls, which 
He will not leave in continual ease, nor in everlasting
sorrow, ordering the one and the other 
according to His wisdom. The Almighty does 
not allow us to enjoy quiet without a change 
that we may not become more careless; nor 
does He keep us ever in sorrow that we may not 
fall in despair. Since we are occupied with this 
question, let us decide to wait for our Heavenly 
Father's own time for putting an end to troubles, 
while we ourselves will pray and lead a life of 
devotion : because to turn to righteousness and 
live in the faith belongs to our obligation, but 
to quiet our sorrow is the work of God. God 
who is mightier than you, who are afflicted in 
temptation, desires to quench that fire, but He 
waits for your salvation. Therefore, as sorrow 
is begotten from ease, likewise we should wait 
for sorrow to give us quiet. It is not always 
winter, nor always summer ; the tempest does 
not always blow, nor does the quiet always last; 
it is not always night, nor all one day. Thus 
also with us when we are sad, a change comes, 
we feel lighter, our hope is stronger — if we pray 
aud in time of sorrow continually thank God. 

And so let us enclose ourselves on all sides 
with truly good and charitable works, and 
thereby be saved from the anger of God. Let 
us make the members of our body the organs 
of righteousness ; let us teach the whole body 
to serve only the cause of virtue. Then we will 
be delivered from present dangers, appease the 
Most High, and reach that inexpressible bliss, 
of which may we all be made worthy by the 
Grace and love for us of our Lord Jesus Christ, 
through Whom be glory to the Father with the 
Holy Ghost, now, ever and forever. Amen. 

\chapter{The Necessity for Divine Revelation, and the Indications of a Revealed Religion}

There are many religions in the world, and each people separately seem to be convinced that they hold the true faith, which was established by God. Why is it that the different races of mankind consider their faith to be the one Divinely revealed? It is because God has revealed Himself to the first people created by Him; then, during the course of time, He revealed Himself to the better people. informing them, and other through them, of His holy will. Fifteen centuries before the birth of Christ, during the time of Moses, God made known His holy will, through Balaam the prophet, who lived in the midst of pagans. About this time, or somewhat earlier, God appeared--for his presence individually may be revealed--to Job and his friends. But sooner than this, when nations began to leave God, and corrupted, by injurious notions, the revealed religion,--then it was that evil spirits commenced to deceive them with their revelations. Finally, there were persons, also, who, abusing the confidence and belief of others in Divine revelation, falsely, but craftily, set up themselves as messengers from God.

Why did God, from the beginning, reveal himself to people? Because, as we have said, the natural knowledge of the human mind concerning God, even before the fall of man, has not the strength, the precision, and the completeness; yet man is responsible for his actions: he is obliged to answer for his attitude toward God, to other people, and to himself. Therefore all people in general believe only in a revealed religion. Some people foolishly say, God has given man a mind, and that is enough forh im. But has not man, even besides his mind, many other teachers, in the persons of his parents, teachers, and guardians? Why should it be against reason to have as our teacher God Himself in those things which pertain to God?

Further, experiences shows that mankind is in a disordered condition, out of which it cannot help itself. Is it natural that people, who are considered educated, while not believing in Divine revelation, should, in matters of knowledge concerning the Divine, separate into parties which are opposed to one another?

\begin{enumerate}
    \item A certain party, in defiance of sound reason, says that God, and His law given to people, do not exist; but they suppose that matter, with its invisible power, (nevertheless, it is supposed by them that matter has its invisible power and law,) has existed forever, and that the world and our living is not controlled by a reasonable Lawgiver.
    \item There are some which say that everything in the world is God; all the things which we can see, they say, is the expression of the invisible soul of the world, which is unconscious, and gets to know itself by an evolution in the different forms of life. These s-called theosophists thereby ascribe to God Himself all the failures, defects, and crimes which proceed from people's abuse of liberty. Other faults of these false teachings have been pointed out before.
    \item Yet there are others, who recognize God as an infinite, \textit{perfect spirit}, the Creator of the universe; but they do not consider Him as the provider of the world, claiming that God, having reated the world, gave it at the time, once for always, wise laws, and that he does not any longer concern Himself immediately about the world, but, say they, it governs itself by the laws given to it.
\end{enumerate}

There have been, and there are such people still, who worship, not one, but many gods, and, at that, not only good ones, but also evil ones, which are opposed one to another. To what inconsistencies does not the fallen reason come, when left to itself in matters which ertain to the knowledge of God! And it thus continues for many thousand years, and it seems there will be no end to the disputes, in deciding the most important questions concerning God and the world, among the wise ones of this world; because to the natural powers and mind of man there is a fixed limit. Take for instance the natural sciences; i.e., the different studies about the visible world.

Let man perfect the instruments necessary for examining the things of this earth; let him discover new powers and laws which were not known to him before; let him enlarge the astronomical lens, and see new starry worlds,--yet in all this he will see only matter with force in action, although according to wise laws, nevertheless according to laws of necessity; and one learned man, such as Newton, will pay homage at the very name of the Creator, while another, such as Leland, will sacrilegiously declare that he did not see God, even through the telescope. What do the sciences say of man? Should we not take the words of Moses concerning the beinning of the human race as Divinely revealed, then, for the want of other historical monuments, the beinning of mankind and the first ages would be covered under darkness of the perfectly unknown, and there would be no end to the conjectures and disputes concerning the beinning of people, the differences of race and languages, etc.

Further, can the science of the soul explain and decide these questions: How is the soul born of the souls of parents? How does the soul act upon the body? and how is it that the soul influences the body?

God Himself, to ends which are most wise, has ordered it so that the soul's action upon the body, and that of the body upon the soul, and the connection between them, should remain beyond the reach of our self-conscience. If man does not know how it is that the soul influences the body, then how may he expound and define the questions, How did the infinite, Divine Spirit bring forth this visible, material world? and how does He act upon it? Therefore the sound reason of nations calls for a faith, and demands a revealed religion; and God actually revealed Himself to people. which for them was and is necessary, as they cannot attain to such knowledge by their own limited powers. Amazingly much is the feebleness of mind in defining the highest questions of knowledge. Not less feeble is the fallen will in accomplishing good and unselfish works without special help from God. This has always been recognized by the best representatives of mankind. St. Paul the Apostle has written: \quote{For the good which I would, I do not; but the evil which I would not, that I practice. For I delight in the law of God after the inward man; but I see a different law in my members, warring against the law of my mind, and bringing me into captivity under the law of sin.}{Rom. vii. 19,22,23.} The philosopher Seneca, who lived in the same age, askes, ''What does it mean? for when we desire one thing, we are drawn by a something to another.'' Another learned pagan, Ovidius, of the same century, has written: ''We always strive after that which is forbidden, desiring the prohibited. I see and value the better, but I follow the worse.'' Thus, as there is a God, it is beyond doubt, also, that His revelation and help to man, in regard to the knowledge of the Divine, and in striving to please God, is necessary, and we have strong reasons to show that, of all religions, the Christian belief is the only one true faith, established by God Himself for the salvation of people; and the Christian may say with the holy apostle Paul, \quote{I thank my God by Jesus Christ; the law of the spirit of life in Christ Jesus has liverated me from the law of sin and death.}{Rom. vii. 25; viii. 2.}

What are the proofs for the Divine character of the Christian religion They are of two kinds. Some are contained in the investigation of the qualities of the Christian religion. The Christian doctrine is the puresst and most elevated in comparison with the teachings of other beliefs. In other religions the ideas of God, and His attributes, and His relations to the world, are not consistent with sound reason, while Christians believe in the Lord God as the most supreme Spirit, with a nature which is most perfect, eternal, everywhere present, most wise, all-knowing, all-powerful, most good, all-holy, all-just, most blessed. Although there are mysteries in the Christian faith, such as are the two principal ones, namely, the trinity of persons in One, of the same Godhead, and the incarnation of the Son of God, which are in most part incomprehensible to the mind; of course it is natural, as God in His substance is incomprehensible; yet they do not contain, in themselves, anything which is contrary to sound reason, but have sides that our mind understands. The Christian doctrine of the future life and the resurrection is also pure and elevated.

The moral law of the Christian religion is so perfact and exalted, that nothing more could be added to it. In regard to God, the law of Christ commands a filial love, which may prompt even the self-sacrifice of one's life for the glory of God, if needs be. Jesus Christ has given us a comandment, by which we are obliged to have a complete love toward our neighbor: \quote{Love your enemies, bless those who do wrong to you and persecute you, that you may be the children of your heavenly Father.}{Matt. v. 44, 45.} \quote{This is my commandment, that you love one another, as I have loved you.}{John xv. 2.} And Christ died for people, when they were sinners,--enemies of God. Finally, in regard to themselves, the law of Christ teaches that Christians must be humble, patient, self-sacrificing.

Still, the proof of the truth of religion because of its elevated teaching about God and the future life, and because of the purity of its moral law, is not yet the final decision, because even the most exalted teaching may be taken as the invention of man.

The most certain and deinite proofs of the Divine character of a fath are the immediate testimonies of God Himself that the belief is true. The testimonies, therefore, must be revelations of a supernatural order, such as the manifestation of the Deity, prophecies, and genuine miracles. When God Himself appears to mankind, saying that this faith is the true one, or tests its truth by miracles, then, of course, we may not doubt it.

But, in other pagan religions, there have been, and yet may be, revelations, prophecies, and miracles worked by evil spirits; therefore the revelations of God, prophecies and miracles also, may not be, as it appears, sure proofs. In this instance, attention must be given to the power, and majesty, and character of the miracles.

Not being able to deny the miracles of Christ, the Pharisees--His enemies--said that He worked miracles, and even cast out devils, by the power of a higher evil spirit. But the Lord answered them, that the kingdom which is divided in itself cannot stand. He cast out thousands, or legions, of demons at once. Moreover, He worked such miracles which could not be performed by evil spirits.

When he gave sight to him that was born blind, the Jews, quarreling among themselves, said: \quote{Can the Devil open the eyes of the blind?}{John x. 21.} Some of the Pharisees themselves said: \quote{A sinful man cannot work such miracles.} Finally, the Lord resurrected the dead, and He Himself rose from the dead and ascended into heaven.

Secondly, where the evil spirits act, there the teaching concerning God is impure, for it is polytheism (like the pagans have) or pantheism, and the moral teaching is defective.

The evil spirits have succeeded in bringing people to deify sins and passions. The ancient nations had gods of wine, adultery, thefy, and other hideous things. Therefore, two kinds of witnesses together are necessary; a faith is true, Divinely revealed, which, in the first place, is holy, pure, and exalted, and opposed to evil spirits, and which, secondly, is proved by revelations and miracles, and by miracles noted for their great power, and of which there are none in other religions; moreover, such ones which put the evil spirits to shame, being driven out of places in which they have ruled.

But, as miracles chiefly demonstrate the truthfulness of a faith, as they serve as witnesses for God Himself, and as unbelievers use all means to overthrow them, it is necessary, therefore, to set our attention upon them as proofs of the Divine character of the Christian religion.

In regards to Divine testimonies or miracles, no religion on earth may compare with Christianity. Miracles, witnessing to the heavenly origin of the Christian faith, come from the very beginning of the human race down to our time.

During 5,500 years--the time of the Old Testament--mankind have been preparing--being educated through supernaturaly revelations and miracles--to receive the Divine Organizer of our faith. Time will not permit to numerate the miracles of ancient times. Just now we regret that we cannot give, as we should, special attention to the great number of Old Testament prophecies concerning Jesus Christ, as applying directly to His person. The miracles of our Lord Jesus Christ, which He worked Himself, have been explained to you on many occasions, and in the future, no doubt, they will continue to be a live source for exhaustless themes of instruction.

We hope to be granted the privilege to explain for you, in a short while, other proofs, demonstrating the truthfulness of the gospels, and finally, with God's help, we will consider Orthodoxy as a sacred distinction in the midst of many Christian professions.

There are such people among Christians who are ashamed of miracles. Such Christians are aashamed of and deny Christ Himself; and He Himself will renounce them before His heavenly Father. Our Lord Jesus Christ, during His life on earth, has often pointed to miracles as to clear and definite proofs of His Divine mission.

While John the Baptists was confined in prison, he sent two of his disciples, for their as well as our benefit, to question Jesus: \quote{Art thou He that cometh, or look we for another? And Jesus answered them, Go and tell John the things which ye do hear and see: the blind receive their sight, and the lame walk, the lepers are cleansed, and the deaf hear,, and the dead are raised up, and the poor have good tidings preached to them; and blessed is he, whosoever shall find none ocasion for stumbling in me.}{Matt. xi. 2,3,4,5,6.}

At another time he said to the Jews, who believed not in him, but among whom there were many who recognized John the Baptist as a saint: \quote{I have witness which is greater than John's; the works which the Father hath given me to fulfill, these same works which I do, bear witness of me, that I was sent by the Father.}{John v. 36; x. 37.}

St John, the Forerunner of Christ, was not granted the power to work miracles, no doubt because the light of Christ must shine for a dark world the clearer of itself.

\textit{Many,} says St. John the Apostle, \quote{came to Jesus and said, that John hath done no miracle.}{John x. 41.}

Of His followers our Lord said, in the last conversation with His disciples: \quote{Believe me that I am in the Father, and the Father in me: or else believe me for the sake of the very works. Verily, verily, I say unto you, he that believeth on me, the works that I do shall he do also; and greater works than thesee shall he do; because I go unto the Father.}{John xiv. 11, 12.}

And before His ascension into the heavens He said to the apostles: \quote{These signs shall follow them that believe: in my name shall they cast out devils; they shall speak with new tongues; they shall take up serpents, and if they drink any deadly thing, it shall in no wise hurt them; they shall lay hands on the sick, and they shall recover.}{Mark xvi. 17, 18.}

O Lord! we are unworthy that Thy wonderful powers be made manifest upon us, as they have been and still do exhibit themselves through They saintly followers. Increase our faith in They works, which Thou hast performed for our salvation, and in the miracles of They saints. Set us aright, O Lord! and save us. Amen.

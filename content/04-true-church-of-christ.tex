\chapter{The True Church of Christ.\footnote{The author is indebted for assistance in compiling the purely theological portions of these papers to the works of the M. Rev. Dr. Sergius, Archbishop of Vladimir, Russia.}}

WHAT is the Orthodox Church? The 
Orthodox Church is a body or community
of people, who, 1 — correctly believe in divine revelation;
and 2 — who obey a lawful hierarchy instituted by our Lord 
Jesus Christ himself, through the holy apostles.
In order to belong to the Orthodox Church two principal conditions
are required: First — to accurately accept,
rightly understand and truthfully confess 
the divine teaching of faith; and secondly — 
to acknowledge the lawful hierarchy or priesthood,
to receive from it the holy mysteries or 
sacraments, and generally to follow its precepts 
in matters concerning salvation. 

Let us now consider the question concerning 
the true and divine doctrine of holy faith. 

The divine teaching of our holy religion is 
contained in the books of the holy scriptures of 
the Old and the New Testaments, and in holy 
Tradition. The principal dogma (these truths), 
i.e. the sacramental ones, (which may also be understood
as the theoretical) are laid down briefly 
in the ''creed,'' which commences with these 
words: \textit{I believe in one God, the Father}, and 
which was compiled by the holy fathers of the 
first two universal councils in the fourth century.
The moral truths of the orthodox faith 
are contained chiefly in the Ten Commandments 
given by God to Moses on Mt. Sinai 
which were completed and explained by our 
Lord Jesus Christ in the Gospel and especially in 
the Lord's Sermon on the Mount. 

The doctrine, which does not agree with the 
true understanding of holy scripture and holy 
tradition, which is preserved in the Orthodox
Catholic Church from the apostles time, is
termed heresy; translated from the Greek
language this word signifies separation. Certainly
it is to be understood that such who separate
or draw others away from the body of the 
church by false teaching, thereby they excommunicate
themselves from her fold. 

Heresy, or injury to the teaching of Christ, 
has begun as early as the times of the apostles. 
St. Paul wrote to Titus, who was bishop on the 
island of Crete: ''A man that is a heretic after 
a first and second admonition refuse, knowing 
that such a one is perverted, and sinneth, being
self condemned.'' (Titus iii : 10, 11). The holy 
apostle Paul has written to the Corinthian 
Christians thus: ''For there must be also heresies
among you, that they which are approved 
may be made manifest among you.'' (1 Cor. ix: 
19). The bishops, as the successors of the 
apostles, endeavored from the earliest times to 
transmit the teaching of Christ, which they received
from the apostles accurately. Thus our 
faith was carefully, even to the letter, transmitted
by tradition. It is plainly understood 
how holy tradition became a channel by which 
truths were conveyed to rising generations, as 
the first bishops themselves received the word 
and also necessary instructions from the 
apostles, not only in writing, but also orally; 
i.e. by word, face to face; therefore it is clear 
that this apostolic tradition was in itself an explanation
of the holy scriptures, as it were — a 
supplement. In regard to holy writ the bishops 
were careful that no false books be counted in 
with the genuine collection, which was left by 
the apostles, and also that the original writings 
of the apostles themselves be not injured or 
marred by heretics through the least addition or 
omission to the text of holy scripture. And 
if a false teacher be found his teachings was at
once examined by the bishops, and they declared
before the Church universal that such and 
such a doctrine was not known to them, that 
they did not receive it from the apostles, and 
that it did not agree with the doctrine of the 
apostles. Heresy caused the gathering of local 
and general councils, in which the false teach- 
ing was compared with the holy scripture and 
tradition and then rejected. In course of time 
the apostolical tradition, which was transmitted 
orally at first, was gradually, as the necessities 
of the church demanded, committed to writing; 
and it is found in the works of the holy fathers 
and teachers of the first several centuries. 

Although all heretics, whoever they be, do not 
belong to the church, yet, judging from the 
character of their false teaching, some are nearer 
to her, while others are greatly separated from 
her, and therefore the church receives into her 
communion the repentant heretics differently; 
namely — by three distinct offices during public 
worship. These offices were formulated in the 
time of the general councils. If we are spared 
we shall in some future time explain these offices,
and also what differences and contradictions
there are in the heresies themselves. Now 
we continue to briefly review the orthodox faith.

The principal dogma of our religion are these: 
1 — The doctrine of God as He is in His being; 
one God in substance, but in three persons; the 
trinity consubstantial and undivided; the Father 
unoriginate; the Son begotten of the Father be- 
fore all ages; and the Holy Ghost, who proceeds 
from the Father. 2 — The doctrine of the Son 
of God, as the Savior of the human race; the 
second person of the Most Holy Trinity, the Son 
of God, who was incarnate for our salvation of 
the Most Holy Virgin Mary, who suffered and 
died in the flesh, arose again; ascended into the 
heavens, and he shall come again to judge the 
living and the dead. 3 — The doctrine of the 
Holy Ghost, as the sanctifier and perfecter of 
the salvation of mankind; that He is sent on 
earth by the Father for the merits of Jesus 
Christ, and abides in the holy, Catholic and 
apostolic church, preserves in her the orthodox 
teaching of faith unimpaired and saves the faith- 
ful chiefly by means of the holy mysteries (or 
sacraments), regenerating, enlightening, edify- 
ing and strengthening in the spiritual life. 
Upon these truths are founded also the other 
dogma of the christian religion; viz: That of 
the Mother of God,\footnote{See the author's Ritual,
Services and Sacraments of the Eastern Apostolic
Church.} the veneration of the saints of God,
sacred images, the administration of the church, etc. 

We have already learned that the true confession
of faith by itself is not sufficient for 
salvation. Of necessity another condition is 
required to belong to the Orthodox Church, and 
that is the recognition of a lawful hierarchy 
(or priesthood), the reception of sacraments 
from the same hierarchy, and obedience to it in 
matters concerning salvation. In a community
of christians in which there is no lawful bishop,
who is the dispenser of the gifts of saving grace,
there are no sacramental gifts of the 
Holy Ghost, there can be no mystery of the body 
and blood of Christ, and where the Holy Ghost 
and Christ are not present, who sacramentally 
abide in christians, there, of course, can be no 
church. Sacred scripture testifies to this very 
decidedly. 

Let us turn our attention to the eighth chapter
of the Acts of the Apostles. What do we 
read there? At the time when a great persecution
arose against the church in Jerusalem aud 
the holy archdeacon Stephen was stoned to 
death, then the christians, excepting the 
apostles, scattered in different places of Judea 
and Samaria. The deacon Philip, who came 
into the city of Samaria, preached Christ there. 
The people with one heart gave heed to what 
Philip said, seeing the miracles which he 
worked; for the unclean spirits came out of 
many; some they left'' with wild cries, and many 
who were impotent and lamed became whole. 
And there was great joy in that city. There 
was a man in that place, one Simon by name, 
who before this practiced sorcery and con- 
founded the people of Samaria, giving himself 
out as some one great. Many followed him, 
saying that he had the power of God. But 
when they believed Philip, who spoke to them 
of the good tidings of the kingdom of God and 
of the name of Jesus Christ, they received baptism
of him, both men and women. And so did 
Simon believe, and after being baptized he remained
with Philip, and, seeing the great 
powers and signs which were manifested, he 
wondered. The apostles, who were in Jerusalem,
having heard that Samaria received the 
word of God, sent to them Peter and John, who, 
having come, prayed over them thai^hey might 
receive the Holy Ghost, and laying their hands 
upon them they received the Holy Ghost. 
Upon seeing that, by the laying on of the 
apostles' hands, the Holy Ghost was given,
Simon brought them money, saying: Give me 
this power, that upon whomsever I lay my 
hands the same will receive the Holy Ghost. 
But Peter said unto him : Thy silver perish 
with thee, because thou hast thought to obtain 
the gift of God with money. Thou has neither 
part nor lot in this matter. 

From this history it can be seen that during the 
time of the apostles there were grades in the 
hierarchy. Philip, who was one of the seven deacons,
notwithstanding that he received grace 
for the office of a deacon from the apostles,
notwithstanding that by the Holy Ghost, who was 
with him, he performed many great works, yet 
he could not bring down the Holy Ghost upon 
the Samaritans, whom he had baptized. But 
when the apostles Peter and John had come 
they prayed and laid their hands upon them. 
Then the Holy Ghost came down upon them 
and was manifested in signs and miracles. The 
apostles transmitted the power of conferring 
the Holy Ghost only to the bishops. In other 
parts of the same book of the Acts of the Apostles,
and in the epistles of St. Paul to Timothy, 
the Bishop of Ephesus, and to Titus, the Bishop 
of Crete, there are plain statements pertaining 
to the grade or office of presbyter, which is a
middle one, between the episcopate and diaconate. 

Which hierarchy is the true and lawful one? 
It is the priesthood which had retained, and 
continued to follow these conditions. 

\begin{enumerate}
    \item In the first place such a hierarchy is true, 
    which received the grace of the Holy Ghost 
    from the apostles themselves in an unbroken 
    line of succession from one to another. If, for 
    instance, in a certain locality the bishops and 
    priests were found to be wanting, the succession
    being broken, and in their absence the 
    laity elected new ones and lay their hands upon 
    them, and proclaimed them to be bishops and 
    presbyters, such a hierarchy would be unlawful
    and without grace, as the laity cannot 
    transmit that which they do not possess themselves 
    — the grace of the priesthood. In the 
    time when the erring church of Rome was the 
    cause of the Protestant separation in the sixteenth
    century, there was not a bishop in any of 
    the countries that sided with them, excepting in 
    England alone,\footnote{Individually we have not the power to assert that 
    the Church of England has retained all the conditions 
    whereby she may not be an erring branch of the Catholic 
    Church.} where protestantism appeared 
    later than in Germany. The Protestants commenced
    to elect and establish presbyters themselves,
    and these ministers not only baptize, 
    but they officiate at a so-called communion 
    service, which of course is not a valid sacrament,
    as the ministers have no apostolic ordination,
    and they are not presbyters. 

    As we learn from history, it is only such a 
    hierarchy which is authentic — that received 
    the grace of the priesthood from the Lord 
    Jesus Christ's apostles themselves, through an 
    unbroken succession of the lawful heirs of this 
    sacrament. And this is necessary. As the 
    inclination to sin is transmitted successively 
    from one to another by inheritance in the con- 
    ception and birth of the body, thus also grace, 
    that is the power of God, which wipes away sin 
    and gives strength in struggle with it, for the 
    merits of the new Adam, the Lord Jesus Christ, 
    being bestowed, it is transmitted uninter- 
    ruptedly by the laying on of episcopal hands in 
    the priesthood, by anointing all christians with 
    holy chrism, and also through sacred acts and 
    visible forms in other sacraments. 

    \item Secondly, an authentic hierarchy is such, 
    which confesses all the truths of boly religion, 
    for there are heresies which entirely deprive 
    bishops and priests of the ministerial grace. 

    \item Thirdly, a priesthood to be lawful must 
    administer the sacraments orderly, according 
    to the rules of the holy church Catholic, not 
    changing essential actions, as there are acts and 
    conditions in the rites of mysteries that are essential,
    without which a certain sacrament may 
    not be valid. Should a sacred minister violate 
    an essential rule he is subject to degradation, if 
    the violation has been intentional, or, at least, 
    the mystery is void of power. The seventh 
    rule of the apostolic canon enjoins: "Should 
    any one, bishop or presbyter, administer not 
    three immersions in baptism in commemoration 
    of the death of the Lord, but one, let him be cast 
    out." And those who were baptized by one 
    immersion, it was ordered that they should be 
    rebaptized. If a priest should consecrate 
    chrism himself, and anoint the newly baptized 
    with it, such an act would not be the mystery of 
    unction with chrism, because it would be the 
    usurpation of the rights and the power of a 
    bishop, and such a thing is forbidden presbyters
    by the sixth rule of the Council of Carthage.
    Should a bishop or priests use only water 
    in place of wine in the mystery of communion, 
    as some heretics do, such an offering would not 
    be a true sacrament. 

    \item Fourthly, to be a lawful and true hierarchy
    the same must be governed and must govern 
    its spiritual charge according to the rules of the 
    holy apostles, the seven ecumenical councils 
    and other laws which are accepted by the 
    Orthodox Church in general. Having apostotized
    from these universal or Catholic church 
    regulations, the Roman church invented a 
    doctrine concerning the supremacy of the 
    Bishop of Rome over all the christian churches. 
    This has been one of the chief causes of the 
    Romish schism or separation from the Orthodox
    Catholic church. 

    \item A fifth condition necessary for proving the 
    lawfulness of the priesthood is its unity with 
    the Orthodox Church in the spirit of peace arnd 
    love. Whoever destroys the unity, except for 
    a genuine and important cause, and the bishops 
    and priests together with christians who follow 
    them, that separate themselves from the higher 
    church authorities, are excommunicated from 
    the church, according to the rules of the 
    apostles and the canons of the councils. 

    The Orthodox Church, which is one, is one 
    spiritual body, animated only by the Holy 
    Ghost, having only one head — the Lord Jesus 
    Christ. 
\end{enumerate}

The Orthodox Church is holy, not having
spot or wrinkle or any such thing (Ephes. v: 
27). She sanctifies sinners by her teaching 
and sacraments. 

The Orthodox Church is Catholic, i.e. collective,
because she was organized by the Lord 
Jesus Christ for the salvation of all people in 
the whole world, and she is the gathering of all 
true believers in all places, times and peoples. 

The Orthodox Church will continue on earth 
until the second coming of Christ, \textit{imperishable 
and not conquered by any powers of hell}. In 
regard to holy doctrine, she is blameless and 
will ever remain unchangeable, as she has 
abiding in her the Holy Ghost, the spirit of truth, 
therefore she is, according to the apostle, \textit{a 
pillar and the foundation of truth} (1 Tim 3: 15). 
The existence of the lawful hierarchy and the 
administration of the holy mysteries will never 
cease in the church. 

The Lord Jesus Christ himself had said: \textit{I 
will build my church, and the gates of hell will 
not prevail against her}, and again: \textit{Behold I 
am with you alway, even unto the end of ages}. 
Therefore, it is the duty of the christian to 
obey the church, for, outside of her, there is no 
salvation. \textit{If thy brother neglect to hear the 
church, let him be to thee as an heathen man
and a publican} (Matt, xviii: 17), saith the 
Lord. 

May God, who is glorified in the Trinity, 
help us by His grace to become, through our 
membership in the church militant on earth, 
members of the church triumphant in heaven, 
that we may glorify His all-honorable and 
majestic name with the angels and saints forever,
without end. Amen.

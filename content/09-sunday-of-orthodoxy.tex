\chapter{Sunday of Orthodoxy}

\textit{Who is so great a God as our God? Thou art the God that alone doest wonders}.

On this first Sunday in Lent, the Church, 
in memory and in thanksgiving for her victory
in the struggles and labors in protecting 
the true Faith against the contentions of evil-minded
heretics, celebrates the ``Triumph of 
Orthodox Christianity,'' and for this reason we 
call this day ``Orthodox Sunday.'' 

It was in 787 A.D. that the Church, in Universal
Council assembled, decreed, among other 
resolutions and canonical acts, that it was lawful 
for Christians to use in their private and public 
worship sacred images, i.e., pictures of our 
Lord Jesus Christ, His Holy Mother, of the 
Saints, and sacred events in Biblical and Christian
history, but, of course, when divine adoration
was ascribed to God alone, and when reverence
is offered in honor of His works -- the 
objects which these cherished pictures represent 
to us. 

The Christian doctrine necessary for our salvation,
as revealed in Sacred Scripture and 
Tradition, has been expounded and delivered 
for us, free from all mixture of human and heretical 
interpretation, by the Seven Great Councils. 
The one mentioned before was the last, namely, 
the Seventh Ecumenical Council. These councils
defined the teaching concerning the Persons 
of the Most Holy Trinity in the one God-head, 
the advent in the world of the Son of God, the 
relations between ourselves and our Savior, 
the relations between the Church Militant on 
earth and the Church Triumphant in heaven, 
the Providence of God in our reward and in 
our punishment, the Apostolic Succession and 
Hierarchical Economy as necessary for the continuance
of Christ's work in the world, the seven 
Sacraments, etc. 

It was not long, however, when again heresy 
began to show itself in some of the branches of 
the Church, and when some ambitious people 
would impose upon the Church their personal 
and fallible opinions. To ward off the false 
shoots and upstarts, and to remind the Christians
of the binding rules and canons of the 
Seven General Councils, a large assembly of 
Holy Fathers and teachers gathered in Constantinople
in 842 A.D., under the protection 
of the good Empress Theodora, and, mindful of 
the Divine Judgment pronounced of the Holy 
Ghost though it was by the condescendingly 
lovable Apostle St. Paul that \textit{if any man love 
not the Lord Jesus Christ let him be Anathema 
Maranatha}, they declared: ``To those who reject
the councils of the Holy Fathers and their 
traditions which are agreeable to Divine Revelation,
and which the Orthodox Catholic Church 
piously maintains, anathema''! 

This council sat in convention during the first 
week of Great Lent. While fasting and praying
they collected all the decisions of the Seven 
General Councils. When Sunday came they 
marched in solemn procession, bearing the holy 
cross, sacred images of our Lord, the Blessed 
Virgin and the Saints, being followed by a multitude
of Christians devoutly chanting under 
the leadership of the learned monk well known 
by the name of \textit{St. Theodore the Studious}, his 
newly composed hymn which you have heard 
today and which translated reads thus: ``To 
Thy most pure Icon (image) we bow down, 
O Blessed One, praying for forgiveness of our 
sins, Christ our God; for, of thine own will, 
thou didst condescend to ascend the cross in
flesh, and thereby to deliver thy creatures from 
the yoke of the enemy. Therefore, we thankfully
cry unto thee, Thou hast filled all things 
with joy, O our Savior, thou who camest to 
save the world.'' 

Having come into the cathedral of St. Sophia 
this religious and noted assemblage offered the 
most impressive praise service, or ``Te Deum,'' 
ever known in the grand liturgies of the Holy 
Orthodox Church. We have in the words of 
the Psalmist David the key-note which re-echoed 
in the thunder of anathemas and resounded in 
the peals of praise of this complete and universal
thanksgiving service: \textit{Who is so great a 
God as our God? Thou art the God that alone 
doest wonders!} Here were recounted all the 
false teachings condemned by the Ecumenical 
Councils, and even persons were anathematized 
for willfully adhering to heresy, who did not 
repent of their sins, and earnestly seek the truth 
by their return to membership in the Church of 
Christ. Among such were those ``who deny 
the existence of God, and unreasonably maintain
that the world existeth of itself, and that 
all things happen through fate and without the 
providence of God; those who insolently dare 
to say that the All-pure Virgin Mary, before her 
bringing forth, in her bringing forth, and after
her bringing forth, was not a virgin; those who 
believe not that the Holy Ghost gave wisdom 
to the Prophets and Apostles, and through them 
proclaimed to us the true way to everlasting 
salvation, and that He confirmed them by wonders,
nor believe that now He dwelleth in the 
hearts of the faithful and true Christians, leading
them into all truth; those who deny the 
immortality of the soul, who reject the councils 
of the Holy Fathers, and the traditions unanimous
with the divine revelation which the Orthodox
Catholic Church with veneration preserveth;
those who defame and blaspheme the holy 
icons which the Holy Church useth to remember
the works of God and of His Saints, so that 
they who look upon the same may be incited to 
fear God and to imitate what they see; and 
those who say the icons are idols.'' 

It may be necessary before we proceed to explain
the word \textit{anathema;} it means condemnation
and excommunication until restored after 
sincere repentance. In some cases it may not 
be only a temporal ban, but a curse. Indeed, 
there are some members of the Church today, 
Christians, who do not fully realize that the 
Church of Christ is a living organism, which, 
through the supernatural indwelling of the 
Holy Spirit, exists as a moral being, empowered
within her sphere not only to bless, but also to 
curse. Such ones of course do not read the 
Bible. Those who studied the Epistles of the 
Apostles know that it was required of the Corinthians
\scripture{to put away from among themselves 
that wicked person}{1 Cor. V: 13} Likewise the 
command was given to Titus, hear: \scripture{A man that 
is an heretic after the first and second admonition
reject}{Tit. III: 10} Did not our Lord 
Jesus Christ say: \scripture{If thy brother neglect to hear 
the Church, let him be to thee as an heathen 
man and a publican?}{Matt. XVIII: 17.} And 
again our Lord speaks: \scripture{Whatsoever ye shall 
bind on earth shall be bound in heaven; and 
whatsoever ye shall loose on earth shall be 
loosed in heaven.}{Matt, XVIII: 18.} 

Since the time of this council which we have 
just now been considering, the Church, annually, 
until our day ``has celebrated the triumph of 
Truth over heresy, and blessed the memory of, 
as well as commended the work of all them that 
by words, writings, teachings, and sufferings, as 
also by a life well-pleasing to God, have contended
for Orthodoxy as her defenders and 
helpers.'' Among those now living are named: 
The Royal and Imperial Benefactors, the Orthodox
Patriarchs of Constantinople, Alexandria, 
Antioch, and Jerusalem, the Holy Synods of the
Russian and other Orthodox Churches, the Most 
Reverend Bishops, the Reverend Clergy, all 
right-believing Christians who, through saving 
faith and good works, are expecting everlasting 
blessedness. Thus the Church today in most 
of the Diocesan cathedrals throughout the 
world, while joyfully praising and honoring 
them that \textit{submitted their understanding to the 
obedience of the Divine revelation}, and have 
contended for the same by following the Holy 
Scriptures and holding fast the traditions of the 
primitive Church, at the same time ``humbly 
supplicates Almighty God for those who, by 
heresy or by schism, have set themselves against 
His evangelical truth that He may soften their 
hearts, open their ears that they may recognize 
His voice, heal their corruptions and deliver 
them out of error.'' 

When we see how the Lord of creation and 
the Shepherd of His elect flock has preserved 
His Church undefiled and whole through long 
ages of the most terrible temptations, and when 
we hear the prophet cry out that \textit{God wills 
no one to be lost, but that all may come to repentance
and to the understanding of truth}, we 
can not else but cry out: \textit{Who is so great a 
God as our God? Thou art the God that alone 
doest wonders}.

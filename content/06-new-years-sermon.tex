\chapter{Sermon On New Year's Day}

Thy Kingdom Come. (Matt. vi:10).

THE first day of a new year may not be kept 
as a holy day and a day of quiet by every 
one, as the Church enjoins that it should be, yet 
it is a day, nevertheless, peculiarly distinguished 
from other days by every thinking man and 
woman. New Year's Day is looked upon by 
some with that awe, which is always respectful 
before the mysterious. For some it has a 
strong fascination, which is expressed in their 
holiday-making, often bordering on senseless 
hilarity. For others it is a short day into 
which they vainly strive to encompass eternity, 
or even the one year which it represents. 

The merry callers, together with the pleasant 
entertainers, and the busy crowd of elders, together
with the happy ones of new and young 
fortune, cannot hide from us even on New 
Year's Day in the great congregation of human 
kind those faces that look on us with serious 
mien, those eyes of careful thought, that wistful
gaze of longing, those eyes that burn with 
a desire. Some of these last named are those 
who were, so to illustrate, moulded into an 
image of melancholy composure, whom painful 
anxiety could not conquer and make of their 
rich natures absolute pessimists, by robbing 
them of their last hope; and some are those 
who have force, power, hidden away down in 
their souls, who persevere, quietly abiding their 
time, when they may openly and fully cherish 
their own desires, satisfy their secret aspirations,
and gain the end of their passion — strong 
ambition. Some again are the ones who very 
nicely put on their back the sheep's skin, but 
inwardly they are the ferocious wolf. They take 
you with their soft hand, but nothing is left 
you; sweetly they look upon you with quiet 
eyes, but you find yourself to be lost; they kiss 
you, and you are betrayed by Judas. Still we 
find among the last mentioned, i. e. of those 
with serious and longing mien, such characters 
as cannot be influenced aside from the path they 
chose for their life walk, either by wealth or by 
social happiness, nor can poverty or misery eat 
and destroy their individuality. Fame, position, 
science, art, comfort and society's opinion call 
out to them : To you will I give all this authority,
and the glory of them \ldots if you will
but worship me, it shall be yours. No, to the 
mighty ones of this world answer they; is it 
right to hearken unto you, rather than unto 
God? This little flock of the chosen ones go 
steadily along the narrow path. Praising the 
Almighty Creator they draw near, and before 
the awful presence of the Supreme Being they 
pray, without condemnation and with boldness 
they dare to say : Father, our Father, who art 
in heaven! hallowed be Thy name, Thy kingdom
come, preserve us from the taint of the 
world, so that the evil spirit with his passions 
and servants may not rule over us; teach us to 
worship Thee in the spirit and in the truth, 
so that the changing etiquette of a vanishing 
sphere, and the vain philosophy of time servers 
may not harm nor forbid us to call to Thee: 
"Lord, Thy kingdom come." 

"I wish you a happy new year." Such is the 
universal greeting on this day among friends. 
Man salutes man on the first day of a new 
year and expresses the hope that the new year 
may be a happy one for each. Ah! and so it is 
happiness, the aim and end of all, which is the 
one thing most desired. That is what christians
ask for when they pray to God: Thy 
kingdom come. And it is just for this purpose
that this altar was built for us. That was the 
desire of our fathers, who contemplated over 
thirty years ago to organize a parish and have 
a house of prayer in this city, and which they 
realized, thanks to the christian sympathy of 
the holy synod of the Kussian church. 

Happiness — that was the mission object of 
the apostles, who walked the earth. It was for 
our happiness that Jesus Christ came and commenced
for the whole world and all time a 
new and everlasting year. Did not even the 
heavens and their spiritual ministers proclaim 
it? Yea, face to face and heart in heart, did 
Mary encompass it. To Joseph it was in a 
dream. It was gloriously indicated to the 
learned magicians by a moving star. But for 
the peasant on the fields the angels sang. Yes, 
for this gift to humankind, for this happiness 
of the new year to the world the spiritual 
powers of heaven thanked the Lord. They 
sang: Glory to God in the Highest, and on 
earth peace, good will toward men. 

As we to-day commence another period by 
which we measure that which we call time, and 
as we feel that this time is gliding past us, 
flowing swifting beyond our reach, and stripping
us, too, of that which we sometimes think
belongs to our person, we surely ought give 
serious thought to the one thing so needful, to 
the happiness we wish our friends from year to 
year, to the great boon our spirits yearn for, 
even though it be on our part sometimes unknowingly.
Let us renew within us the 
faculties of our soul, so overburdened with a 
generally prevailing materialism. Let us renew
within us our hearts, and prepare a clean 
habitation for this great gift. Let us strengthen 
our desire, once elevated, and let us reach out, 
and accept, and follow this great happiness of 
God in man! O God, save us from the rule 
which our own severally different, irreligious 
and selfish opinions create, and from the kingdom
of darkness, and let The kingdom come! 

It is often just so with the life of a man as 
the traveler of great deserts experiences. He
now is under the hot sun with no water, and 
then the cold atmosphere of the night finds him 
without a roof. With sore foot and tired eye 
he goes along until he comes to an oasis; the 
fresh scene dispels the monotony; his heavy 
heart is gladdened. Such an oasis we find even 
in the barren hearts of all men of the world; 
but not so often, not so fruitful and so refreshing
as in the life of an humble and obedient
believer in the Allguiding Providence of God. 
Oh, christians! watch for those bright moments 
in your life. Prosper in the real happiness 
and shine forth in the darkness of a sinful 
world — a light to others. Stop on these green 
and fresh pastures. Best. Look over the past 
and examine the way. Consider the different 
kinds of temptations you underwent. Know 
thyself; where were you the weakest? Which 
place on the road was it the most difficult to 
pass? How have you come out of the battle? 
What is it you have lost? Did you gain anything?
If so, is it good for your salvation? 
Can your neighbor profit by it? 

Let us hear Solomon, the wisest of all earthly 
born, the richest and greatest king of his time; 
let us hear what he says in one of those bright 
moments of his life, when he was most fit and 
capable to rightly diagnose his self-examination. 
He says : "7 have seen all the works that are 
done under the sun; and, behold, all is vanity 
and vexation of Spirit. 1 communed with my 
own hearty saying, Lo, I am come unto great 
estate, and have gotten more wisdom than all 
they that have been before me in Jerusalem; 
yea, my heart hath great experience of wisdom 
and knowledge. And I gave my heart to know 
wisdom, and to know madness, and folly : I perceived
that this also is vexation of spirit. I 
made me great works, I builded me houses, I 
planted me vineyards. All kinds of trees and 
flowers I had in my gardens. I made me pools 
of water. I got me servants and maid servants. 
I gathered me also silver and gold. I was 
great. Also my wisdom remained with me. 
Then I looked on all the works that my hands 
had wrought, and on the labor that I had 
labored to do, and, behold, all was vanity and 
vexation of spirit, and there was no profit 
under the sun * * * When we read farther 
on and come to the close of Solomon's repentant
confession, he says: "Let us hear the conclusion
of the whole matter, fear God and keep 
His commandments, for this is the whole duty 
of man. For God shall bring every work into 
judgment, with every secret thing, whether it be 
good, or whether it be evil." 

On this New Year's Day, when we look over 
the past and see our mistakes, our weakness, 
our folly, and our sins, and* when from to-day 
we look to the future with renewed hope, wishing
as much as ever before — a happy new 
year, our good resolutions must be carried out 
with a strong will. When we learn to seek our
happiness in that one thing above all needful; 
when we learn to bend our wills to the pleasures 
of that Supreme Will, which rules all, then we 
will have found the good portion, which shall 
not be taken away from us. The Kingdom of 
God will have come. 

But to some I know this appears to be a hard 
saying. It is well to talk about such high 
things, but how can we practice a heavenly life 
upon earth, they question. Certainly the 
thought about earthly things is indispensable to 
our earthly life. Do but observe how we 
abandon things heavenly, for things earthly, 
and we shall find it not so difficult to put aside 
earthly things for things heavenly. We limit 
the time we employ in works of charity and 
religious practice, in order to have more time 
for worldly things. Some times we go into the 
Church of God, and at the same time we are 
thinking of that which engages our minds at 
home and at our business. And some times, 
even, while standing bodily in the house of 
prayer, our thoughts are attracted elsewhere, 
by our worldly affections, or by the passions 
which rule in us; even the very prayer of some 
is tainted by flitting worldly thoughts! Now do 
the very reverse. Do that which is necessary
for your earthly existence, but endeavor not to 
extend it beyond the necessary, and strive to 
liberate yourself as much as possible from such 
labor, in order to have more time and freedom 
for works of piety. Kestrain your thoughts 
from earthly things, not only when standing 
before God in His temple, but wherever you 
may be, when obliged to busy yourself with 
earthly things, occasionally turn away your 
thoughts and especially your desires for them, 
and lift up your heart unto heaven and God. 
When you set about worldly affairs, remember 
God, and ask for His blessing and assistance; 
when you go to rest, remember God and give 
thanks unto Him for His assistance in your 
labors, and for the gift of rest. 

Thus we may unite every earthly work, not 
contrary to the law of God, with a love of things 
above, and, so to say, change earthly and visible 
things, into things heavenly and spiritual. 
When thou lookest upon the sun, said once a 
saint, seek the true sun, for thou art blind. 
When thou turnest thy gaze upon light, turn 
towards thy soul, and see whether thou hast 
there the true and blessed light, which is the 
Lord. 

May the light of our Lord Jusus Christ illumine,
may His Spirit strengthen each of us, 
and may our walking according to His Word 
and His Life, lead us all here upon earth to set 
our affections on things above, and thereby conduct
us to the blessed contemplation of Him in 
heaven, where reigneth supreme the happiness 
of all sincere seekers of the true new year. 

Amen.

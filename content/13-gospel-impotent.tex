\chapter{Sermon on the Gospel Relating to the Impotent.}

John v: 15.

JESUS went up to Jerusalem. These are the 
words which begin the Gospel appointed to 
be read at the Liturgy on this Sunday. How 
many thousands must have hurried to those 
feasts in honor of the One irue God, in that 
splendid city of types and symbols of things 
not yet made clear, and to be manifested in the 
midst of pagan and barbarous nations! It must 
have been a great multitude that almost continually
went up to Jerusalem. So did our Lord 
Jesus Christ go with them. He was in the crowd, 
sometimes known, but most often His own received
Him not. Those who went up to Jerusalem
in those days went to the only one Temple, 
where the only divinely authorized priesthood had 
offered the one acceptable type of the all-available
sacrifice. To be understood, in plainer language,
we may say of them that they went to 
church. There is nothing extraordinary; the 
going is a simple fact of going to church. But
how did they go? Absurd? Does it seem to 
you so? Certainly, it seems that most people go 
along in the same way. At the bottom of the 
movement which makes people go somewhere, 
there is an intention, and I think that when people
go to church they go with a fixed purpose. 
If you ask how they go, with the purpose of 
disclosing their one or several intentions, then I 
answer with the word of God as recorded by 
St. Paul : For who among men knoweth the 
things of a man, save the spirit of the man, 
which is in him? Even so the things of God 
none knoweth, save the Spirit of God. But for 
our edification and salvation on this as well as 
on several other memorable occasions, we are permitted
to see and to learn how our Lord Jesus 
Christ went up to Jerusalem, and consequently 
how He went to Church. 

He went to Jerusalem at the feasts in order to 
give a larger number of people an opportunity 
of seeing and hearing the Truth. He went to 
the house of His Father, and openly manifested 
His power as the Son of God. Before the doctors
and lawyers He testified to the Old Testament
prophecies as the Messiah. He passed by the 
way of sufferers so that He might help them, and 
where He was not expected, there He was found.
He gave courage to those who had lost hope, and 
those who hoped on and long He rewarded with 
His grace. Where an angel by stirring the 
water cured a sick one in a year, He, the Lord 
of Angels, heals both body and soul, be it of 
one, or a hundred, or a thousand, and now as 
He did then and ever will as long as one be 
found who will surrender himself as others have 
done : Lord, thou canst make me whole! 

But let us return to Bethesda, not because it is 
the public hospital where lay a multitude of the 
needy, but because, as impotent folk ourselves, 
we will find there on this occasion the bread of 
life and the very source of the never-ending 
stream of living water. Let us follow Christ 
through the five porches, where a certain man 
icas which had been thirty and eight years in 
his infirmity. When Jesus saw him lying, and 
knew that he had been now a long time in that 
case, He saith unto him : Wouldst thou be made 
whole? The sick man answered Him: Sir, I 
have no man, when the water is troubled, to put 
me into the pool; bid while I am coming another 
steppeth down before me. 

Why is it that Jesus passed by all the others 
and stopped by this one ? He did so in order to 
show His power and also His love for mankind:
His power, because the disease had become incurable
and the weakness of the sick one was 
beyond hope; His love for mankind, because 
the Provider and Merciful One, in preference to 
others, looked upon such who were especially 
worthy of pity and charity. Let us not quickly 
pass by this place without giving our attention 
to the thirty-eight years during which this sick 
one continued in his weakness. Let all who 
struggle with continual poverty, or pass their 
lifetime in sickness, or those who find themselves 
in difficult and threatening circumstances, or 
cast down in the storm and tempest of sudden 
troubles, let them all hear of it. No one can be 
so faint-hearted, so mean and unfortunate as not 
to bear all that happens to us manfully and 
with all cheerfulness, when looking upon this 
special one at the watering place of Bethesda. 
If he suffered for twenty years, or ten, or only 
five years, would they not be sufficient to break 
the strength of his soul? But he remains in 
that condition for thirty-eight years, and does 
not break down in spirit, but shows great 
patience. Hear his wisdom, for indeed a Christian
may lend ear to this sick one's philosophizing.
Jesus came up and said to him: Wouldst thou 
be made whole? Who does not know such
a thing? Why of course the impotent desired 
to become well. Then for what reason does he 
ask? Certainly not because of ignorance: for 
Him who knows the secret thoughts of people, 
that which was apparent and open to all was to 
Him the more so simple. Why does He ask? 
As He said to the centurion: I will come and 
heal him, not because He did not know at first 
what his answer would be, but because foreseeing 
and knowing well the answer He wished to 
give the centurion the opportunity to disclose 
before all his piety, which was concealed as 
if under a shadow, and to say: Lord, I am 
not worthy that Thou shouldst come under 
my roof. So also this impotent, of the nature 
of whose answer He knew, the Lord asks does 
he wished to be healed, not because He knew 
that not himself, but for the purpose of giving 
the sick one a chance to speak out of his misfortune
and to become a teacher of patience. 
If the Savior had healed this man silently, we 
would have lost much by not learning of the 
strength of his soul. 

Christ does not rule the present only, but He 
offers to the future also His condescending and 
great care. In the impotent He showed us a 
teacher of patience and courage for all times to 
come, having put him to the necessity of answering
the question: Wouldst thou be made whole? 
But what of him? The impotent was not 
offended; he did not become angry; he said not 
to the inquirer: Thou seest me infirm, thou 
knowest that my sickness is of long standing, 
and thou inquirest, do I desire to regain my 
health? But did you come to laugh at my misfortunes
and to make light of other people's 
troubles? You know how faint-hearted the sick 
become when they lay in bed even one year; but 
whose sickness lasts for thirty-eight years, does 
it not become natural for such a one to lose all 
better knowledge, wasting away in the course of 
so long a time? The impotent, however, said 
nothing like this, nor even thought of it, but he 
answered with much modesty : Sir, I have no 
man, when the water is troubled, to put me into 
the pool. See how many troubles unitedly 
grieved this man : his disease, his poverty, and 
the absence of a lifting, a helping hand. While 
I am coming, another steppeth down before me. 
This is that which is the saddest of all, and it 
ought to have softened even a stone. Does it 
not seem as if you can see this man each year 
creeping along until he has crept up to the very 
side of the pool, then stopping each year just 
before the reach of a bright hope? And it is 
the more burdensome, as he experienced this not
for two, or three, or ten, but for thirty-eight 
years. He did everything in his power, but did 
not obtain the result; the labor was accomplished, 
but the reward for labor went to another one 
during all these many years; and, what is more 
burdensome, he saw others healed. 

The good fortune of others around us compels 
us to plainly see in the contrast our own misfortune; 
it was the same then with the impotent. 
Nevertheless, for so long a time he struggled 
with sickness, with poverty, and with loneliness, 
seeing that others were healed, while he himself, 
although he always tried, but never could reach 
his desire, and not hoping in the future to liberate
himself of suffering, with all this against 
him he did not retreat, but renewed his endeavor 
each year. And we, if we once pray to God and 
do not obtain what we have asked for, we immediately
become disappointed and fall into extreme 
carelessness, so that we stop praying and lose 
fervor. May we according to worth praise the 
impotent, as we may in the same way condemn 
our negligence? What justification and forgiveness
may we expect when he was patient for 
thirty-eight years, while we become despondent 
so soon. To this one it has been said: Arise, 
take up thy bed and walk. And if we be or be 
not infirm in body, or soul, mind, character, or
condition, it is in every instance demanded of us 
all to take up our cross and to follow Him, our 
Lord Jesus Christ, with whom, after a faithful 
following, we shall see ourselves gloriously resurrected.

Because Jesus did these things on the Sabbath
the Jews persecuted Him. When persecuted
for doing the works which proclaimed Him 
to be the Redeemer of the world, did our Lord 
justify Himself before His enemies and prove 
the Divine right of His most exalted mission? 
If He did not, then how could we, even to this 
day, hope in our salvation? Let us see how 
Jesus, on this occasion, has justified Himself; 
for the manner in which He proves His innocence
shows us whether He belongs to the number
of such who are ruled, or to the free, to 
those who serve, or whether He is of those who 
command. His action seemed to be a great 
iniquity, a sin against the law; for he who once 
gathered wood on the Sabbath was according to 
the law stoned to death for carrying a burden on 
the Sabbath (Num. xv: 32, 36). Now Christ was 
accused of the same crime, namely, that He did 
not keep the Sabbath. Does He ask forgiveness 
as a servant and as a man under subjection, or 
does He appear as one who has power and independence
as a Master, who is above the law and
who Himself giveth the commandments? How 
does He justify Himself? My Father, says He, 
worketh even until now, and I work. Do you 
see His might? If He was lower or lesser than 
the Father, then what He had said would not be 
counted in His acquittal, but it would be to a 
greater accusation and to a greater condemnation.
If one does something which is lawful to 
be done only by one who is above Him, and then 
having been taken and given to judgment, he 
says : as another higher one has done so, I also 
have done so, he would not only free himself 
from the charge in this way, but he would subject
himself to a greater accusation and sentence, 
because to take upon one's self that which is 
above one's dignity can be done only by a self-conceited
and proud person. Therefore, if Christ 
was lower than the Father, then what He had 
said would not be to His justification, but to a 
greater condemnation. But as He is equal with 
the Father, there is no fault in the words of 
Jesus Christ. 

To understand the better what has been said, 
let us remember that His disciples had once 
broken the Sabbath in the field by pulling ears 
of corn and eating them; now He violated it 
Himself; the Jews accused them, and now they 
accuse Him. We will now investigate as to how 
He clears them and how He justifies Himself, so 
that we may learn from the difference between 
one and the other of the superiority of His 
justification. How did He justify His disciples? 
Have you not read what David did when he 
hungered? (Matt, xii: 3.) Defending creatureservants,
He calls to mind David, a fellow-servant
like unto themselves, but justifying Himself 
He reaches out with His speech to the Father: 
My Father workeih, aud I work. Perhaps some 
one might ask : What kind of work does Jesus 
speak of, if after six days God rested from all 
His works? (Gen. ii: 2.) It is the everyday, the 
continual guidance and providence, for God not 
only made all nature, but He also keeps His 
creation. Do you refer to the angels, the archangels,
or to the higher powers, and, in a word, 
to all things visible and invisible? Yes, all are 
under His providence, and if it would go outside 
the realm of His activity, then it falls to pieces, 
becomes destroyed and would perish. And so 
our Lord, desiring to show that He is the provider
and not the object of providence, the 
worker and not the object of activity, He has 
therefore said : My Father worketh, and I work, 
thereby proving His equality with the Father, 
and to whom with the Holy Ghost be all glory 
now and to ages of ages. Amen.

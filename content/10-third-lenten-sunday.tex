\chapter{Sermon Preached on the Third Sunday of Great Lent.}

Whosoever will come after me, let him deny himself, and take up his cross and follow me. (Mark viii. 34).

THIS third Sunday of Great Lent is the first 
day of the Holy Cross week. In the midst 
of this holy season the Church allows her children
to taste of the sweetness of the tree. In 
the course of his forty days' journey over the 
solitary wilderness of penance, the weary wayfarer
comes to a tree, its shade inviting, its fruit 
beneficial. He throws off his burden, with composure
and confidence he nestles at the foot of 
the wood, and is refreshed. And this is just 
what the Church of God prepares for the sincere
followers of Jesus Christ. Were we capable 
of understanding the Almighty's plan, carried 
out by the Church for our salvation, what consolation
and benefit, both spiritual and temporal,
would be ours. 

Whosoever will come after me, let him deny 
himself, and take up his cross and folloio me. 
This is what Jesus said to the multitude that
followed him. They followed Him, some from 
curiosity, anxious to see a miracle performed, 
others in earnest, eager to listen to the words 
that came from the lips of Him, who taught as 
one having the power, while there were those 
which followed in humble obedience, grateful 
tor the charity bestowed on them, or on their 
loved ones, by the Prophet of Nazareth, and, 
alas! there were some that dogged every step of 
this Good Shepherd, who led his human flock 
over the green hills of Galilee, or quenched 
their thirst with living water down in the valley 
of Jordon, spies — which were to take Him and 
put Him to a horrible death — nailing Him hands 
and feet to a crossed wood. As he turned toward
the people He addressed them, saying: 
For what shall it profit a man if he shall gain 
the whole world and lose his own soul. Whosoever
shall be ashamed of me and of my words 
in this adulterous and sinful generation, of him 
also shall the Son of Man be ashamed when he 
cometh in the glory of his Father with the holy 
angels (Mark viii: 36, 38). 

Jesus has nowadays many followers desirous 
of consolation, but few of tribulation. All desire
to rejoice with Him, few are willing to 
endure anything for Him. Many reverence His
miracles, few follow the ignominy of His cross. 
What cross is it which our Lord would have 
us bear? Hardships, sickness, slander, persecutions,
poverty, desertion of friends, the heavy 
cares of public responsibility, yea — and death 
itself. We may for a time be forsaken of God; 
sometimes we are troubled by our neighbors, 
yea — and by those whom we love dearly; and 
what is more, oftentimes we may even become 
weary of ourselves. If Christ bore the cross 
for all mankind, Christians are expected to help 
carry the cross of at least some of their neighbors.
But how? By bearing in patience the 
failings and weakness of our neighbor. By not 
becoming ill-tempered when a brother or sister 
sets forth his or her opinion as to this or that. 
By hushing the serpent's hiss of envy, and 
showing sympathy and gladness when one either 
above or beneath us proves himself worthy of 
public praise, though we ourselves may not be 
so much as noticed. By denying ourselves the 
wicked pleasure of making jest of a soul which 
goes about acting strangely, and especially when 
we do not understand nor see plainly the results 
of such conduct. By denying ourselves the 
luxuries which may supply the want of many 
who suffer misery.
Whosoever will come after me, let him deny 
himself, and take up his cross and folloiv me 
(Mark viii: 34). O bow powerful is the pure 
love of Jesus, which is mixed with no self- 
interest, with no self-love ! Are not those to be 
called mercenary who are ever seeking consolations?
The Holy Fathers of the Church tell 
us that " such are lovers of themselves, but not 
of Christ. Where shall one be found, the Holy 
Fathers continue, who is willing to serve God 
for naught?" In their great fervor to serve, in 
their deep and vast love for God, the Holy 
Fathers had not noticed how they themselves 
were growing into perfection by following our 
Lord Jesus Christ. Yes, in their persons and 
lives we have many types and good examples 
unto the salvation of our souls; yea, and unto 
the salvation of the world. And so we must 
lose all that which unites us to this world that 
is passing away. If we link our life with and 
make it one with that of the common animal, 
subject our reason to vain pursuits and the better
qualities of our spirit to passions of the flesh, 
then we lose our life, for dust will go to dust. 
But ivhosoever shall lose his life for the sake 
of Christ and the Gospel, the same shall save it; 
i.e., he who is dead to self, when one becomes
separated from the principles which make one 
long for the pleasure and desire of this world's 
life, in which there is no thought for eternity, 
the same shall save his life, though he seemingly
perish in discomfort and suffer banishment
at the hands of a self-idolizing society, for 
he lives through Christ and the Gospel with the 
spiritual life which never grows old and is everlasting.

Daring the whole of this week the Church is 
continually reminding us of the cross. And 
many a weary soul is sighing for rest. . . . 
Shall my burden be lessened? There seems to 
be for some no end to sickness. The cares of 
duty are constant and heavy. There are those 
which cannot find a friend who could understand
their inmost soul and soothe their troubled 
conscience. Oh! that God would give me that 
inward peace, they cry without faith and in despair.
Deny yourself all passions which are 
prompted by selfish motives and by interests, 
which will pass away as the light of day is lost 
in the darkness of night; deny these passions 
all gratification whatever. What a hard lot ! 
you might say. The circumstances surrounding
our lives are very pressing. . . . What will 
people say? We must keep up with the rest of 
the world. We are in a pitiable condition. No
one should criticise us. We should be left to 
do as we please. But the Church says we must 
give up the pleasure of being even a little 
ambitious. Yes, the Church of Christ says: 
Forsake all false ambition. Christians, be not 
discouraged. The Infinite Wisdom itself watches 
over your salvation. In selfishly moping over 
our own woes, over our little cross, have we forgotten
the Cross of Christ? The Church, then, 
will remind us of it. 

This week is set aside for the worship of the 
cross. Here is the tree of life. The cross that 
our Lord Jesus Christ carried to Mount Calvary 
and made it an altar, on which He offered himself
up to God the Father as a sacrifice. Let us 
come to the shade of this tree and rest. This wood 
has been planted for our own benefit. Let us in 
holy meditation bring to our mind the suffering 
of that bleeding form outstretched above us, 
although it is difficult and for some impossible 
to feel for one moment the anguish of that 
cross, borne all the weary way from Bethlehem; 
then our little crosses, which we have merited 
by our sins, will not be a yoke of thorns, but 
an altar on which we may offer up to God our 
love. Our course is not finished. The road lies 
before. . . . Lent is still in season. Now, while 
we enjoy the protection of the cross, let us also
supply ourselves with strength, i. e., the Grace 
of God, for the journey is not finished and 
the way is so uncertain. Let us refresh and 
strengthen ourselves with a supply of the fruit 
of the wood. The fruit is the flesh and blood 
of our Saviour, who was sacrificed in order to 
appease the righteous wrath of the Infinite God 
for the sins of all mankind, beginning with the 
disobedience of Adam. Shall not our Creator 
receive us when we humbly and gratefully come 
to Him together with His infinitely beloved, 
His Only-begotten, as the Light which is of 
Light, His Son Jesus Christ? 

Christians, ye who come to this tree, go not 
away without tasting of its sweet fruit. Our 
Holy Church brings us this week to the cross 
to be refreshed by renewing our spirit, so that 
our own cross be not a burden but a blessing. 
Do we forget our duty towards our Mother 
Church? Yes, unfortunately some do. May we 
not let go unheeded the advice of the parent, 
which has given us birth in baptism for a new 
life, but come to the foot of the cross and cast 
off our yoke of sins, be absolved of all that 
which is impure and wicked, either in thought, 
or in desire, or in deed, by confession, and in 
holy communion with Jesus Christ be reconciled
to God. Amen.

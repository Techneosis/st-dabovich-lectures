\chapter{Sermon on the Feast of the Ascension}

\footnote{This sermon was written and delivered by the
author a few days after reading a most beautiful, but
lengthy, sermon in the Russian language by the celebrated
Metropolitan Philaret of Moscow.}
And while they looked steadfastly toward heaven as He 
went up, behold, two men stood by them in white apparel, 
which also said: Ye men of Galilee, why stand ye gazing 
up into heaven? (Acts I: 10, 11). 

THE "two men in white apparel," who immediately
after the ascension of the Lord appeared
to the Apostles and asked them why they 
stood gazing up into heaven, were without doubt 
themselves inhabitants of heaven; therefore it 
is not to be supposed that this was displeasing 
to them, or that they desired to direct the gaze of 
those men of Galilee elsewhere. No. They desire
only to put an end to the inert amazement 
of the Apostles when saying: Why stand ye 
gazing up into heaven? Having aroused them 
from their amazement, they draw them into meditation,
and teach them and us with what thoughts 
we should gaze into heaven, following our Lord 
Jesus who hath ascended thither. This same
JesiiSy they added, which is taken up from you 
into heaven, shall come in like manner as you 
have seen Him go into heaven. 

The disciples of the Savior then beheld the 
exact fulfillment of His words which Mary 
Magdalene had recounted to them: I ascend 
unto my Father and your Father, and to my 
God and your God. They could not but conclude
that those joyful visitations which He had 
bestowed upon them during the forty days after 
His resurrection from the dead, those instructive 
conversations with Him, that palpable communion
between them and His divine humanity, were 
at that moment ended. When neither hand nor 
voice could any longer reach Him, they followed 
Him with their eyes, eager to detain Him; they 
looked steadfastly toward heaven as He went 
up. We can conceive what an immeasurable 
bereavement the Apostles must have felt after 
the ascension into heaven of Jesus, who was all 
and everything in the world to them; and it is 
this very bereavement for which the heavenly 
powers hasten to console them when telling them 
that this same Jesus . . . shall come. 

In considering the circumstances of the ascension
of Christ into heaven, we may first note the 
blessing which He then gave to the Apostles,
and it come to pass, says the Evangelist Luke, 
while He blessed them He was parted from 
them and carried up into heaven. What an endless
current of the grace of Christ is thus revealed
unto us, Christians ! The Lord begins a 
blessing, and before its completion ascends into 
heaven; for while He blessed them He was carried
up into heaven. Thus even after His ascension
does He still continue invisibly to impart 
His blessing. It flows and descends continuously
upon the Apostles; through them it is diffused
upon those whom they bless in the name 
of Jesus Christ; those who have received the 
blessing of Christ through the Apostles spread 
it among others; and thus do all who belong to 
the Holy Catholic and Apostolic Church become 
partakers of the one blessing of Christ. As the 
dew of Hermon that descended upon the mountain
of Zion, so does this blessing of peace descend
upon every soul that riseth above passions 
and lusts, above vanity and the cares of the 
world; as an indelible seal does it stamp those 
who are of Christ in such a manner that at the 
end of the world He will by this very sign call 
them forth from the midst of all mankind, saying,
Come ye blessed! 

And now, my brethren, let us consider how
needful it is for us to endeavor to gain now and 
to preserve this blessing of the Ascended Lord, 
which descends upon us also through the Apostolic
Church. If we have received and preserved 
it, we shall, at the future advent of Jesus Christ, 
be called together with the Apostles and the 
saints to participate in His kingdom: Come ye 
blessed! But if, when He shall call the blessed 
of His Father, this blessing either be not found 
in us, or we be found in possession only of the 
false blessing of men who themselves have not 
inherited the blessing of the Heavenly Father 
by grace and in the sacraments, then what will 
become of us? Yea, I say, let us consider this 
vital point before the opportunity be taken away. 
The day of the Lord cometh as a thief in the 
night. From this same unexpectedness of His 
second coming our Lord Himself draws for us 
Christians a saving warning : Watch, therefore, 
for ye know not what hour your Lord doth come. 
Do not be led away by curiosity or credulity, 
and beware of such ones who pretend to know 
more than Christ hath granted them to know. 
Let us endeavor rather to know what failings we 
have, to number our transgressions, and to seek 
a limit to them in repentance. Let us take heed 
lest the children of this world and our own passions
lull our spirits into sleep, till the approach 
of that longed for, yet dreadful hour: When 
the Lord come. 

The blessing of the Lord come upon you by 
His grace and love towards man, always, now, 
and ever and unto the ages of the ages. 

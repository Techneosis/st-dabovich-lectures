\chapter{Sermon to Those Preparing for Holy Communion.}

Let a man examine himself, and so let him eat of that 
bread, and drink of that cup. For he that eateth and 
drinketh unworthily, eateth and drinketh damnation unto 
himself. (1 Cor. xi: 28, 29.) 

THESE words of the Apostle are terrible, 
and so are they as the truth unfailing. 
Verily the judgment is heavy and the damnation 
terrible for the one who receives the body and 
blood of Christ without due honor and without 
such a disposition of the spirit as is required. 
A terrible judgment has befell that apostate 
people who sentenced our Lord Jesus Christ to 
be crucified, when they cried: His blood be on 
us and on our children (Matt). But that unfortunate
people did not know the mystery of the 
incar iation of the Son of God, and committed 
that greatest crime in blindness and ignorance: 
For had they known, says St. Paul, they should 
not have crucified the Lord of Glory. Of how 
much sorer punishment, suppose ye, shall be 
worthy of such an one, who was born in Christianity,
from childhood taught in the mysteries 
of the faith, and notwithstanding all this, hath — 
by light-heartedness and carelessness — trodden 
underfoot the Son of God, and hath counted the 
blood of the covenant, wherewith he was sanctified,
an unholy thing, and hath done despite 
unto the spirit of grace (Heb. x: 29). 

Hence the reason why the Church with such 
care strives to prepare us for the reception of 
the life-giving mysteries of Christ by fast, prayer, 
repentance. After a few moments when the cup 
of the covenant shall be brought out, unto which 
we must approach in order to reanimate within 
us, renew and strengthen our covenant with 
Jesus Christ, we will hear the last call of the 
Church which summons thus: With fear of 
God and faith approach ye. In those sacred 
moments let be hushed within us all other 
thoughts, let be banished from our souls all 
other feelings, besides those unto which the 
Holy Church would elevate our spirits. Let us 
draw near with fear of God, faith and love, 
that we may be partakers of the life eternal. 

That we may inspire within us that sacred 
fear, let us consider: Where are we now? Before
whom do we stand? Unto what do we approach?
Where are we? Moses, Moses, called 
God to His selected leader of Israel, draw not 
nigh hither : put off thy shoes from off thy feet, 
for the place whereon thou standest is holy
ground (Ex. iii: 5). Since the place unto which 
God once descended has become sanctified, and 
to which the man who was called the friend of 
God could not approach without care, then how 
much holier is the place which is sanctified by 
such often repeated descensions of the Holy 
Ghost at the consecration of the terrible mysteries
upon which even the angels look with fear. 

Before whom do we stand? It is the God of 
unapproachable glory, from whose presence it 
was once that Mt. Sinai blazed and trembled; 
the God Almighty, who spake and it was done; 
He commanded, and it stood fast; that which is 
not, He nameth a thing existing; He maketh to 
die and maketh to live; He lowereth unto hell 
and raiseth up again; the God All-holy, who 
bears not with iniquity and shuns unrighteousness;
the Lord, a jealous God, who exacts of 
children the sins of fathers even unto the third 
and fourth generations; the God All-righteous, 
who came down to see the wickedness of the 
citizens of Sodom and Gomorrah, and which 
cities the heavenly flame swallowed up. It is 
true that God appears to us here in His body 
and blood, without external grandeur and glory, 
without terrible manifestations; for, were it 
otherwise, we would say as the Israelites had
said: Let not God speak with us, lest we die. 
(Exod). 

Unto what do we approach? To the Divine 
Body, which Simeon, the saintly old man, had 
ouce received in his hands with holy fear; to 
the Divine Body, by the touch of which the sick 
were healed, the leprous cleansed, and which 
the demons feared; before the nakedness and 
wounds of which the sun darkened, the earth 
quaked, the rocks brake; to the most Glorious 
Body, which ascended into the heavens and 
upon which the Cherubim and the Seraphim 
look with fear. True it is that it appears to us 
in the form of common food, but were it otherwise
we would say with Peter : Depart from me, 
for I am a sinful man, O Lord! (Matt, v: 8). 
And so it is here we stand, such is the presence 
we stand before, and such is that unto which we 
approach. Great is the gift we receive from the 
hand of the Lord. Holy is His most pure body, 
holy is His life-creating blood, and therefore let 
us approach the cup of the covenant with greater 
care and more fear that we may not be scorched 
with its flame, that we may receive the flesh and 
blood of Christ not unto judgment and condemnation,
but unto the cleansing, the sanctification,
and the enlivening of our nature decaying
in sin. Amen.

\chapter{Gospel of the Prodigal}

\reference{First read: Luke XV. 11-32.}

You have heard today's gospel. The 
parable of the ``prodigal son'' is not a thing 
new to you. You have heard of, you have seen 
the prodigal sin, and fall low, down, deep into 
all the consequential miseries of iniquity. If 
you do not know, you have heard of the boundless
mercy of a pitying God. You may understand
how a good father takes back to his heart 
his beloved child, once lost, but found again. 
You know the parable of the prodigal son. 
Then why is it that the church, year after year, 
recalls to our memory this parable? She does 
so in order to strengthen us in the way of salvation.
Until we have passed the final limit, and 
receive our sentence at the hands of the Divine 
Judge, we belong to the Church Militant, i.e. 
while we are on earth, we are obliged to continually
struggle for the good. 

In order to obtain conscientious peace, love,
spiritual prosperity, and holiness, we must 
always battle with the evil. The more high 
and purely spiritual the condition is, which we 
strive to attain, the more fierce is the fight, and 
our warfare must be constant with wrong, infidelity,
superstition, prejudice, and corruption. 
Indeed, we must overcome ourselves, we must 
get the better of self. 

You, no doubt, have seen men and women 
wasting their living in the most hideous visible 
form of sin, but dare you stand in the awful 
presence of the Most Pure Being and Creator, 
and say that you are not a prodigal? Do you 
not wish to come back to God -- the Heavenly 
Father? Sinners, yes, we are sinners! One of 
the greatest meditators on the ways of Divine 
Providence -- the Prophet-King David -- in his 
confession to God, says: ``\textit{Thy commandment 
is very broad.}'' And, in this light, there is not 
a commandment, or a law, which we have not 
transgressed. 

Surely my time was not spent with harlots, 
some might say, but did you make careful use 
of your time, which is not yours, for it belongs 
to Him who gave it, and did you without wasting, 
treasure it so that it now bears a hundred-fold 
of profit, pleasing to the Receiver of virtue in
abundance? I had never lost control of myself, 
so that, by unawares, my table prove to be a 
scanty board of husks, and my companions a 
herd of swine. Yet, you may not assert that 
you beautified your soul with a holy character, 
nor did you enrich your intellect with an everlasting
wisdom; and your heart, is it clean, does 
it expand so that the Holy Ghost freely makes 
His abode there? Does it know the needy and 
the deserving? Does it go out toward its neighbor,
yearning to share with its very existence -- 
giving up all self-interest, and even the comforts
of an earthly life? 

We, all of us, make up one household. We 
are members of one and the same family. And 
you will never taste of true happiness, nor 
know what it is to be blessed, until you have 
learned this lesson. You may be a younger 
son or daughter, but if you be the prodigal, 
remember, that in your \textit{Father's House there is 
bread enough, and to spare; come to yourself;} 
consider your life, remember the free and confiding
innocence of your first youth, now that 
you are firmly fastened, and yet lost, look within 
and find yourself. And when you have found 
yourself, you will easily find God. He will see 
you, while you are \textit{coming, yet far off, your
Father, and He will be moved with compassion, 
He will fold you in His arms and kiss you}.

If you are not a younger member of the 
family, may you not be the elder son of the 
Father? You may not be a lavish spendthrift, 
nor a wanderer, and you may enjoy the quiet 
of home, but are you secure? \textit{The enemy may 
sow tares in the field of your heart, while you 
are comfortably asleep}. You may live in your 
father's house faithfully and continually, you 
may have the oversee of all the work, and the 
servants, yet are you secure? No, if you do not 
give yourself concern of the whereabouts of 
your younger brother, you are not secure for 
all time. You may be the oldest, you may 
know all the secrets of the household, the keys 
of all chests and doors may be in your possession,
you may live in the Grace of God, and 
enjoy the light of your Heavenly Father's 
countenance, still, remember the elder son in the 
parable! For the want of charity for :m erring 
one, a sinner, an inexperienced one, for one 
who labored under a wrong opinion, he -- the 
heir and firstborn -- came in danger of losing 
all at the end. He was the cause of much 
anxiety to his father who came out and entreated
him. This one's pride (a false pride it
was) that suffered. The father had to reason 
with his son, who thought his sense of justice 
was being injured. In the absence of virtue, 
and charity -- the principal one -- we see the 
elder son blind to his own condition, for he 
dared to assert his rights, while justice belonged 
to the real owner, his loving Father. 

And now, my brethren, if we be the elder 
members of God's family, let us think of the 
responsibility, and not fall from Grace, but continue
in His House. To the young, and to the 
prodigal, if their conscience be not yet lost, the 
Divine voice calls, come to your Father, and 
tell Him all, He waits with open heart. Amen.
